
\documentclass[class=report,crop=false]{standalone}
\usepackage[screen]{../exo7book}

\begin{document}

%====================================================================
\chapitre{Fonctions de plusieurs variables -- 2ème année -- A. Hanani}
%====================================================================

% A virer

\newcommand {\e}{{\rm e}}
\newcommand{\ja}{\mathrm{J}}
\newcommand{\jac}{\mathrm{Jac}}
\renewcommand{\d}{\,\mathrm{d}}
\newcommand {\Rrot}{{\rm rot}}
\newcommand{\aire}{{\rm Aire}}
\newcommand{\vol}{{\rm Vol}}
%\newcommand {\s}{\mathbb{S}} %sphere





\tableofcontents




\chapter{\bf Fonctions de plusieurs variables}
\section{Généralités}

\thispagestyle{empty}

\vskip4mm

\noindent On se place dans l'espace vectoriel $\Rr^n$. Le produit scalaire usuel de $x=(x_1,\dots ,x_n)$ et $y=(y_1,\dots ,y_n)$, noté $x.y$ ou $\langle x,y\rangle$, est défini par
$$x.y=x_1y_1+\dots +x_ny_n.$$
La norme euclidienne sur $\Rr^n$ est la norme associée à ce produit scalaire. Pour $x\in \Rr^n$, la norme euclidienne de $x$, notée $\| x\|$, est définie par
$$\Vert x\Vert =\sqrt{x.x}=\sqrt{x_1^2+\dots +x_n^2}.$$

\vskip4mm

\begin{definition}Soit $A=(a_1,\dots ,a_n)$ un point de $\Rr^n$. On appelle boule ouverte de centre $A$ et de rayon $r>0$, notée $B_r(A)$, l'ensemble suivant :
$$B_r(A)=\left\{M\in \Rr^n\mid \|M-A\|<r\right\}.$$
\end{definition}

\vskip4mm

\noindent Remarquez que, si $M=(x_1,\dots ,x_n)$ alors $\|M-A\|=\sqrt{(x_1-a_1)^2+\dots +(x_n-a_n)^2}$.

\vskip6mm

\begin{definition} Soient $U$ une partie de $\Rr^n$ et $A\in U$.
\begin{enumerate}
\item On dit que $U$ est un voisinage de $A$ si $U$ contient une boule ouverte centrée en $A$.
\item On dit que $U$ est un ouvert de $\Rr^n$ si $U$ est un voisinage de chacun de ses points.
\end{enumerate}
\end{definition}

\vskip4mm

\noindent{\bf Exemples. }L'ensemble vide, $\Rr^2$, tout rectangle ouvert $]a,b[\times ]c,d[$ et tout disque ouvert de $\Rr^2$ sont des ouverts de $\Rr^2$.

\vskip6mm

\begin{definition}
Un sous-ensemble $U$ de $\Rr^2$ est dit simplement connexe s'il n'a pas de trous. Autrement dit, si son bord $\partial U$ est connexe (formé d'un seul morceau).
\end{definition}

\vskip4mm

\noindent{\bf Exemples. }Un disque de $\Rr^2$, un rectangle de $\Rr^2$, $\Rr^{+*}\times \Rr$ et $\Rr^2$ sont simplement connexes.

\vskip6mm

\begin{definition} Soit $n,m\in \N^*$. Une fonction $f$ de $n$ variables à valeurs dans $\Rr^m$ est une application d'un domaine $D$ de $\Rr^n$, appelé domaine de définition de $f$, dans $\Rr^m$. On la note 
$$\begin{array}{ccccl}f&:&\Rr^n&\to &\Rr^m\\ &&x&\mapsto &\left(f_1(x),f_2(x),\dots ,f_m(x)\right).\end{array}$$
Le domaine de définition ${\mathscr D}_{f_i}$ de $f_i$ est le sous-ensemble de $\Rr^n$ donné par
$${\mathscr D}_{f_i}=\{(x_1,\dots ,x_n)\in \Rr^n\mid f_i(x_1,\dots ,x_n)\in \Rr\}$$
et le domaine de définition ${\mathscr D}_f$ de $f$ est ${\mathscr D}_f={\mathscr D}_{f_1}\cap {\mathscr D}_{f_2}\cap \dots \cap {\mathscr D}_f{_m}$.
\end{definition}

\vskip4mm

\noindent Nous nous limiterons souvent aux dimensions $2$ et $3$, la généralisation aux dimensions supérieures ne posant pas de problème particulier. Voici quelques exemples simples.

\vskip4mm

\begin{enumerate}
\item Distance d'un point à l'origine en fonction de ses coordonnées :
$$\begin{array}{ccccl}f&:&\Rr^2&\to &\Rr\\ &&(x,y)&\mapsto &\sqrt{x^2+y^2}.\end{array}$$
\item Surface d'un rectangle en fonction de sa longueur et sa largeur :
$$\begin{array}{ccccl}f&:&\Rr^2&\to &\Rr\\ &&(x,y)&\mapsto &xy.\end{array}$$
\item Surface d'un parallélépipède en fonction de ses trois dimensions :
$$\begin{array}{ccccl}f&:&\Rr^3&\to &\Rr\\ &&(x,y,z)&\mapsto &2(xy+yz+xz).\end{array}$$
\item Surface et volume d'un parallélépipède en fonction de ses trois dimensions :
$$\begin{array}{ccccl}f&:&\Rr^3&\to &\Rr^2\\ &&(x,y,z)&\mapsto &\left(2(xy+yz+xz),xyz\right).\end{array}$$
\item Coordonnées polaires d'un point du plan :
$$\begin{array}{ccccl}f&:&\Rr^+\times [0,2\pi[&\to &\Rr^2\\ &&(r,\theta)&\mapsto &(r\cos \theta ,r\sin \theta ).\end{array}$$
\end{enumerate}

\vskip6mm

\begin{definition} Soit $f$ une fonction de $2$ variables. Le graphe ${\mathscr G}_f$ de $f$ est le sous-ensemble de $\Rr^3$ formé des points de coordonnées $(x,y,f(x,y))$ avec $(x,y)\in {\mathscr D}_f$. Cela se note :
$${\mathscr G}_f=\{(x,y,z)\in \Rr^3\mid (x,y)\in {\mathscr D}_f\mbox{ et }z=f(x,y)\}.$$
\end{definition}

\vskip4mm

\noindent{\bf Commentaire. }Représenter graphiquement le graphe n'est possible que pour les fonctions d'une seule variable ou deux variables. Dans le second cas, le graphe est une surface de l'espace $\Rr^3$ et comme on ne sait pas faire des dessins dans l'espace, on se ramène à ce qu'on sait faire : c'est à dire faire des dessins dans un plan.

\vskip6mm

\begin{definition}Soit $f$ une fonction de deux variables. La ligne de niveau $z=a\in \Rr$ est
$$\displaystyle L_a=\left\{(x,y)\in {\mathscr D}_f\mid f(x,y)=a\right\}$$
et la courbe de niveau $z=a$ est la trace de ${\mathscr G}_f$ dans le plan $\{z=a\}$ : 
$${\mathscr G}_f\cap \{z=a\}=\{(x,y,a)\in \Rr^3\mid f(x,y)=a\}.$$
\end{definition}

\vskip4mm

\noindent{\bf Remarque. }On obtient la courbe de niveau $a$ en remontant à l'altitude $a$ la ligne de niveau $a$. En dessinant un nombre suffisant de courbes de niveau, on obtient une idée de l'allure de la courbe. Pour plus de précision, on peut aussi tracer, pour quelques valeurs de $a$, les graphes des fonctions partielles
$$f_1:x\mapsto f(x,a)\quad \mbox{et}\quad f_2:y\mapsto f(a,y).$$
La première représente l'intersection du graphe avec le plan $y=a$ et la seconde représente l'intersection du graphe avec le plan $x=a$.

\vskip6mm

\noindent{\bf Exemple. }Soit $f$ telle que $f(x,y)=x^2+y^2$. La ligne de niveau $L_c$ est vide si $c<0$, se réduit à $\{(0,0)\}$ si $c=0$, et c'est le cercle de centre $(0,0)$ et de rayon $\sqrt{c}$ si $c>0$. On remonte $L_c$ à l'altitude $z=c$ : le graphe est alors une superposition de cercle de centre $(0,0,c)$ et de rayon $\sqrt{c}$, $c>0$. Il prend l'allure suivante :

%\vskip2mm
%
%\centerline{
%\begin{pspicture}(-10,-1.5)(4,4)
%\psset{unit=0.7}
%\psset{viewpoint=20 30 10,Decran=100}
%\psSolid[object=grille,ngrid=0.2,base=-1 1 -1 1,linecolor=gray,action=draw*](0,0,0)
%\defFunction[algebraic]{f_3}%
%(t){0.5*cos(t)}{0.5*sin(t)}{0.25}
%\defFunction[algebraic]{f_4}%
%(t){0.7*cos(t)}{0.7*sin(t)}{0.49}
%\defFunction[algebraic]{f_5}%
%(t){0.9*cos(t)}{0.9*sin(t)}{0.81}
%\defFunction[algebraic]{f_6}%
%(t){1.2*cos(t)}{1.2*sin(t)}{1.44}
%\psSolid[object=courbe,action=draw,linewidth=0.04,r=0,linecolor=black,resolution=360,
%function=f_3,base=0 pi 2 mul]%
%\psSolid[object=courbe,action=draw,linewidth=0.04,r=0,linecolor=black,resolution=360,
%function=f_4,base=0 pi 2 mul]%
%\psSolid[object=courbe,action=draw,linewidth=0.04,r=0,linecolor=black,resolution=360,
%function=f_5,base=0 pi 2 mul]%
%\psSolid[object=courbe,action=draw,linewidth=0.04,r=0,linecolor=black,resolution=360,
%function=f_6,base=0 pi 2 mul]%
%\axesIIID[linewidth=1pt,linewidth=1pt,linewidth=1pt,linecolor=black,labelsep=2pt](0,0,0)(1.4,1.4,1.9)
%\psline{->}(4,2)(5,2)
%\psdot(0,0)
%\end{pspicture}
%%
%\begin{pspicture}(-3,-1.5)(10,4)
%\psset{unit=0.7}
%\psset{viewpoint=20 30 10,Decran=100}
%\psSolid[object=grille,ngrid=0.2,base=-1 1 -1 1,linecolor=gray,action=draw*](0,0,0)
%\defFunction[algebraic]{f}(u,v){u*cos(v)}{u*sin(v)}{u^2}
%\psSolid[object=surfaceparametree,function=f,base = 0 1.2 0 pi 2 mul, ngrid=30,incolor=gray,fillcolor=red,linewidth=0.5\pslinewidth]
%\axesIIID[linewidth=1pt,linewidth=1pt,linewidth=1pt,linecolor=black,labelsep=2pt](0,0,1.4)(1.4,1.4,1.9)
%\end{pspicture}
%}
%
%\vskip4mm

\section{Limite et continuité}

\vskip4mm

\noindent La notion de limite et continuité de fonctions d'une seule variable se généralise en plusieurs variables sans complexité supplémentaire : il suffit de remplacer la valeur absolue par la norme euclidienne.

\vskip6mm

\begin{definition}Soit $f:\Rr^n\to \Rr$ une fonction définie au voisinage de $A\in \Rr^n$, sauf peut \^etre en $A$.
\begin{enumerate}
\item On dit que $f$ admet $l$ comme limite au point $A$, on écrit $\displaystyle \lim _{M\to A}f(M)=l$, si
$$\forall \varepsilon >0,\; \exists \alpha >0,\; \forall M\in \Rr^n,\; 0<\|M-A\| \leq \alpha \rightarrow |f(M)-l|\leq \varepsilon .$$
\item On dit que $f$ admet $+\infty$ comme limite au point $A$, on écrit $\displaystyle \lim _{M\to A}f(M)=+\infty$, si
$$\forall \varepsilon \in \Rr,\; \exists \alpha >0,\; \forall M\in \Rr^n,\; 0<\|M-A\| \leq \alpha \rightarrow f(M)\geq \varepsilon .$$
\item On dit que $f$ admet $-\infty$ comme limite au point $A$, on écrit $\displaystyle \lim _{M\to A}f(M)=-\infty$, si
$$\forall \varepsilon \in \Rr,\; \exists \alpha >0,\; \forall M\in \Rr^n,\; 0<\|M-A\| \leq \alpha \rightarrow f(M)\leq \varepsilon .$$
\end{enumerate}
\end{definition}

\vskip4mm

\noindent Pour calculer les limites, on ne recourt que rarement à cette définition. On utilise plut\^ot les théorèmes généraux : opérations sur les limites et encadrement.

\vskip6mm

\begin{theoreme}[\bf Opérations sur les limites] Soient $f,g:\Rr^n\to \Rr$ définies au voisinage de $A\in \Rr^n$. Si $f$ et $g$ admettent des limites en $A$, alors
$$\begin{array}{ll}\displaystyle \lim _{A}(f+g)=\lim _{A}f+\lim _{A}g\;,&\quad \displaystyle \lim _{A}(fg)=\lim _{A}f\times \lim _{A}g\; ,\\ \\ \displaystyle \lim _{A}\frac{1}{g}=\frac{1}{\lim _{A}g}\qquad \mbox{ et }&\quad \displaystyle \lim _{A}\frac{f}{g}=\frac{\lim _{A}f}{\lim _{A}g}.\end{array}$$
\end{theoreme}

\vskip4mm

\noindent{\bf Remarque. }Les résultats ci-dessus sont aussi valables pour des limites infinies avec les conventions usuelles :
$$l+\infty =+\infty,\quad l-\infty =-\infty,\quad \frac{1}{0^+}=+\infty ,\quad \frac{1}{0^-}=-\infty ,\quad \frac{1}{\pm \infty }=0,$$
$$l\times \infty =\infty \; (l\neq 0),\; \infty \times \infty =\infty \mbox{ (avec règle de multiplication des signes).}$$
Les formes indéterminées sont : $\displaystyle \frac{0}{0}$, $\displaystyle \frac{\infty }{\infty}$, $\displaystyle 0\times \infty$, $\displaystyle 1^{\infty}$, $0^0$ et $+\infty -\infty $.

\vskip6mm

\begin{theoreme}[\bf Théorème d'encadrement] Soient $f,g,h:\Rr^n\to \Rr$ trois fonctions définies au voisinage de $A\in \Rr^n$ telles que $f\leq h\leq g$. Si $\displaystyle \lim _{A}f=\lim _{A}g=l$ alors $h$ admet une limite au point $A$ et $\displaystyle \lim _{A}h=l$.
\end{theoreme}

\vskip4mm

\noindent Si une fonction admet une limite en un point, celle-ci est unique. D'o\`u le résultat qui suit.

\vskip6mm

\begin{proposition} Soit $f:\Rr^n\to \Rr$ une fonction définie au voisinage de $A\in \Rr^n$, sauf peut \^etre en $A$.
\begin{enumerate}
\item Si $f$ admet une limite $\ell$ au point $A$, alors la restriction de $f$ à toute courbe passant par $A$ admet une limite en $A$ et cette limite est $\ell$.
\item Par contraposée, si les restrictions de $f$ à deux courbes passant par $A$ ont des limites différentes au point $A$, alors $f$ n'admet pas de limite au point $A$.
\end{enumerate}
\end{proposition}

\vskip6mm

\noindent Un point $M(x,y)$ du plan peut \^etre repéré par ses coordonnées polaires $(r,\theta)$ dans un repère dont l'origine est au point $(a,b)$. Elles sont reliées aux coordonnées cartésiennes par les relations :
$$x=a+r\cos \theta \quad \mbox{et}\quad y=b+r\sin \theta \qquad \mbox{avec }\; \theta \in [0,2\pi [.$$
Or $r=\sqrt{(x-a)^2+(y-b)^2}$ donc $(x,y)$ tend vers $(a,b)$ si, et seulement si, $r$ tend vers $0$.

\vskip6mm

\begin{theoreme}[\bf Calcul de limites en coordonnées polaires]Soit $f:\Rr^2\to \Rr$ une fonction définie au voisinage de $(a,b)\in \Rr^2$, sauf peut \^etre en $(a,b)$. Si
$$\lim _{r\to 0}f(a+r\cos \theta ,b+r\sin \theta )=\ell \in \Rr$$
existe indépendamment de $\theta$, alors $\displaystyle \lim _{(x,y)\to (a,b)}f(x,y)=\ell$.
\end{theoreme}

\vskip6mm

\noindent{\bf Exemple 1. }{\it La fonction $f$ admet-elle une limite au point $(0,0)$ dans les cas suivants ?}
$$\begin{array}{ll}\displaystyle f(x,y)=\frac{x^3}{x^2+y^2},&\displaystyle f(x,y)=\frac{xy^2}{x^2+y^2},\\ \\ \displaystyle f(x,y)=\frac{xy}{x^2+y^2},&\displaystyle f(x,y)=\frac{xy^2}{x^2+y^4}.\end{array}$$

\vskip6mm

\begin{definition} Soit $f:U\to \Rr$, o\`u $U$ est une partie de $\Rr^n$. On dit que $f$ est continue au point $A\in U$ si $\displaystyle \lim _{M\to A}f(M)=f(A)$. On dit que $f$ est continue sur $U$ si $f$ est continue en tout point de $U$.
\end{definition}

\vskip4mm

\noindent{\bf Commentaire. }Pour étudier la continuité d'une fonction de plusieurs variables, le principe est le m\^eme qu'en une seule variable. On décompose la fonction proposée en somme, produit, quotient et composée de fonctions connues pour \^etre continues.

\vskip6mm

\noindent{\bf Exemple 2. }{\it Etudier la continuité de la fonction $f$ dans les cas suivants :}
$$1.\; f(x,y)=\left\{\begin{array}{cl}\displaystyle \frac{xy^2}{x^2+y^2}&\mbox{si }(x,y)\neq (0,0)\\ \\ 0&\mbox{si }(x,y)=(0,0).\end{array}\right. \quad 2.\; f(x,y)=\left\{\begin{array}{cl}\displaystyle \frac{xy}{x^2+y^2}&\mbox{si }(x,y)\neq (0,0)\\ \\ 0&\mbox{si }(x,y)=(0,0).\end{array}\right.$$

\vskip8mm

\section{Dérivées partielles premières}

\vskip4mm

\begin{definition}Soit $f:U\to \Rr$ une fonction définie sur un ouvert $U$ de $\Rr^2$. La dérivée partielle première de $f$ par rapport à $x$ au point $(a,b)\in U$ est définie, si elle existe, par
$$\frac{\partial f}{\partial x}(a,b)=\lim _{x\to a}\frac{f(x,b)-f(a,b)}{x-a}.$$
De fa\c con analogue, la dérivée partielle première de $f$ par rapport à $y$ au point $(a,b)$ est définie, si elle existe, par
$$\frac{\partial f}{\partial y}(a,b)=\lim _{y\to b}\frac{f(a,y)-f(a,b)}{y-b}.$$
\end{definition}

\vskip4mm

\noindent{\bf Remarque. }Les définitions sont similaires si $f:U\to R$, o\`u $U$ est un ouvert de $\Rr^3$. Par exemple, la dérivée partielle première par rapport à $x$ au point $(a,b,c)\in U$ est définie, si elle existe, par
$$\frac{\partial f}{\partial x}(a,b,c)=\lim _{x\to a}\frac{f(x,b,c)-f(a,b,c)}{x-a}.$$

\vskip6mm

\begin{definition}Soit $\vec{v}$ un vecteur non nul. La dérivée directionnelle de $f$ suivant le vecteur $\vec{v}$ au point $A$ est définie, si elle existe, par
$$D_{\vec{v}}f(A)=\lim _{t\to 0}\frac{f(A+t\vec{v})-f(A)}{t}.$$
\end{definition}

\vskip4mm

\noindent{\bf Remarques. }
\begin{enumerate}
\item Si $\vec{v}=\vec{i}=(1,0)$, on retrouve $\displaystyle \frac{\partial f}{\partial x}(A)=D_{\vec{i}}f(A)$.
\item Si $\vec{v}=\vec{j}=(0,1)$, on retrouve $\displaystyle \frac{\partial f}{\partial y}(A)=D_{\vec{j}}f(A)$.
\end{enumerate}

\vskip6mm

\noindent{\bf Exemple. }{\it Soit $f$ la fonction définie sur $\Rr^2$ par
$$f(x,y)=\frac{x^3+y^3}{x^2+y^2}\;\mbox{ si }(x,y)\neq (0,0)\quad \mbox{et}\quad f(0,0)= 0.$$
\begin{enumerate}
\item Etudier l'existence des dérivées partielles premières de $f$ au point $(0,0)$.
\item Etudier l'existence de la dérivée de $f$ suivant un vecteur non nul au point $(0,0)$.
\end{enumerate}
}

\vskip4mm

\noindent \underline{Solution}. \begin{enumerate}
\item On a : $f(x,0)=x$ et $f(0,y)=y$. Donc
$$\lim _{x\to 0}\frac{f(x,0)-f(0,0)}{x}=\lim _{x\to 0}\frac{x}{x}=1\quad \mbox{et}\quad \lim _{y\to 0}\frac{f(0,y)-f(0,0)}{y}=\lim _{y\to 0}\frac{y}{y}=1.$$
Ainsi $\displaystyle \frac{\partial f}{\partial x}(0,0)$ et $\displaystyle \frac{\partial f}{\partial y}(0,0)$ existent et on a : $\displaystyle \frac{\partial f}{\partial x}(0,0)=1$ et $\displaystyle \frac{\partial f}{\partial y}(0,0)=1$.
\item Pour tout vecteur $\vec{v}=(v_1,v_2)$ non nul, on a :
$$\lim _{t\to 0}\frac{f(tv_1,tv_2)-f(0,0)}{t}=\frac{v_1^3+v_2^3}{v_1^2+v_2^2}.$$
Donc $f$ admet une dérivée suivant tout vecteur non nul au point $(0,0)$.
\end{enumerate}

\vskip6mm

\begin{definition}On suppose que $f$ admet une dérivée partielle première par rapport à chaque variable en tout point de $U$. On appelle dérivée partielle première de $f$ par rapport à $x$ (resp. $y$) la fonction définie sur $U$ par 
$$\displaystyle \frac{\partial f}{\partial x}:(x,y)\mapsto \frac{\partial f}{\partial x}(x,y)\quad \left(resp.\ \frac{\partial f}{\partial y}:(x,y)\mapsto \frac{\partial f}{\partial y}(x,y)\right).$$
\end{definition}

\vskip6mm

\noindent{\bf Méthode. }Pour calculer une dérivée partielle par rapport à une variable, il suffit de dériver par rapport à cette variable en considérant les autres comme des constantes paramétriques.

\vskip6mm

\noindent{\bf Exemple. }{\it Calculer les dérivées partielles premières des fonctions $f_i$ définies par}
$$f_1(x,y)=x^2y^3\e ^{xy},\qquad f_2(x,y,z)=\e^z\cos (x^2+y^2).$$

\vskip8mm

\section{Différentielle et jacobienne}

\vskip4mm

\begin{definition}On suppose que $f:\Rr^2\to \Rr$ admet des dérivées partielles premières par rapport à chaque variable au point $(a,b)\in \Rr^2$. On appelle différentielle de $f$ au point $(a,b)$ l'application linéaire de $\Rr^2$ dans $\Rr$ qui à $(h,k)$ associe :
$$\frac{\partial f}{\partial x}(a,b)h+\frac{\partial f}{\partial y}(a,b)k.$$
\end{definition}

\vskip4mm

\noindent{\bf Notation. }En physique, on interprète $h$ et $k$ comme de petites variations des variables $x$ et $y$ et on les note plut\^ot $\d x$ et $\d y$. En fait, les applications linéaires de $\Rr^2$ dans $\Rr$ définies par
$$\d x:(h,k)\mapsto h\qquad \mbox{et}\qquad \d y:(h,k)\mapsto k$$
forment une base de $\mathscr{L}(\Rr^2,\Rr)$ et la différentielle de $f$ au point $(a,b)$ s'écrit dans cette base :
$$\frac{\partial f}{\partial x}(a,b)\d x+\frac{\partial f}{\partial y}(a,b)\d y.$$

\vskip2mm

\noindent En notant $\d f$ la différentielle de $f$, ceci justifie l'écriture abrégée suivante :
$$\d f=\frac{\partial f}{\partial x}\d x+\frac{\partial f}{\partial y}\d y.$$
Elle signifie
$$\forall (x,y)\in U,\;\; \d f_{(x,y)}=\frac{\partial f}{\partial x}(x,y)\d x+\frac{\partial f}{\partial y}(x,y)\d y.$$

\vskip6mm

\noindent{\bf Exemple. }Soit $f$ la fonction définie sur $\Rr^2$ par $f(x,y)=\sin x\e ^{2y}$.

\vskip3mm

\noindent On a : $\displaystyle \frac{\partial f}{\partial x}(x,y)=\cos x\e ^{2y}$ et $\displaystyle \frac{\partial f}{\partial y}(x,y)=2\sin x\e ^{2y}$. La différentielle de $f$ est définie  par
$$\forall (x,y)\in \Rr^2,\;\; \d f_{(x,y)}=\cos x\e ^{2y}\d x+2\sin x\e ^{2y}\d y.$$
Par exemple, au point $(0,0)$, on aura : $\d f_{(0,0)}=\d x$.

\vskip6mm

\noindent{\bf Matrice jacobienne. }Si les $m$ fonctions coordonnées de $f:\Rr^n\to \Rr^m$ admettent des dérivés partielles premières par rapport à chaque variable en un point $A\in \Rr^n$, la différentielle de $f$ au point $A$ est une application linéaire de $\Rr^n$ vers $\Rr^m$. Sa matrice par rapport aux bases canoniques de $\Rr^n$ et $\Rr^m$ est la jacobienne de $f$ au point $A$. Pour plus de clarté, on donne la définition en dimension réduite.

\vskip6mm

\begin{definition}Soit $f:U\to \Rr^2$, o\`u $U$ un ouvert de $\Rr^3$, telle que
$$\forall (x,y,z)\in U,\; f(x,y,z)=(f_1(x,y,z),f_2(x,y,z)).$$
On appelle matrice jacobienne de $f$ au point $(a,b,c)$ la matrice des dérivées partielles premières de $f_1$ et $f_2$, lorsqu'elles existent :
$$\ja _f(a,b,c)=\left(\begin{array}{ccc}\displaystyle \frac{\partial f_1}{\partial x}(a,b,c) &\displaystyle \frac{\partial f_1}{\partial y}(a,b,c)&\displaystyle \frac{\partial f_1}{\partial z}(a,b,c)\\ \\ \displaystyle \frac{\partial f_2}{\partial x}(a,b,c) &\displaystyle \frac{\partial f_2}{\partial y}(a,b,c)&\displaystyle \frac{\partial f_2}{\partial z}(a,b,c) \end{array}\right).$$
On appelle différentielle de $f$ au point $(a,b,c)$ l'application linéaire $\d f_{(a,b,c)}$ dont la matrice dans les bases canoniques de $\Rr^3$ et $\Rr^2$ est $\ja _f(a,b,c)$.
\end{definition}

\vskip4mm

\noindent{\bf Exemple 1. }Coordonnées polaires d'un point du plan :
$$\begin{array}{ccccl}f&:&\Rr^+\times [0,2\pi[&\to &\Rr^2\\ &&(r,\theta)&\mapsto &(r\cos \theta ,r\sin \theta ),\end{array}\quad \ja _f(r,\theta)=\left(\begin{array}{cc}\cos \theta &-r\sin \theta \\\displaystyle \sin \theta &r\cos \theta \end{array}\right).$$

\vskip4mm

\noindent{\bf Exemple 2. }Soit $f:\Rr^3\to \Rr^2$ l'application définie par $\displaystyle f(x,y,z)=(\e^{xy},z\sin x)$. La matrice jacobienne de $f$ existe en tout point $(x,y)\in \Rr^2$ et est donnée par :
$$\ja _f(x,y,z)=\left(\begin{array}{ccc}y\e^{xy}&x\e^{xy}&0 \\\displaystyle z\cos x &0&\sin x \end{array}\right).$$

\vskip4mm

\noindent{\bf Exemple 3. }Soit $f:\Rr^2\to \Rr^3$ l'application définie par : $\displaystyle f(x,y)=(x^2+y^2,\e^{xy},x+y)$. Alors la matrice jacobienne de $f$ au point $(x,y)$ est :
$$\ja _f(x,y)=\left(\begin{array}{cc}2x&2y\\y\e^{xy}&x\e^{xy}\\ 1&1\end{array}\right)$$

\vskip4mm

\noindent{\bf Complément. }Soit $f:U\to \Rr$, o\`u $U$ est un ouvert de $\Rr^3$, telle que les trois dérivées partielles premières au point $A\in U$ existent. Alors
$$\ja _f(A)=\left(\begin{array}{ccc}\displaystyle \frac{\partial f}{\partial x}(A)&\displaystyle \frac{\partial f}{\partial y}(A)&\displaystyle \frac{\partial f}{\partial y}(A) \end{array}\right)$$
et, pour tout $\vec{v}=(v_1,v_2,v_3)$,
$$\d f_A(\vec{v})=\frac{\partial f}{\partial x}(A)v_1+\frac{\partial f}{\partial y}(A)v_2+\frac{\partial f}{\partial z}(A)v_3=\left(\begin{array}{ccc}\displaystyle \frac{\partial f}{\partial x}(A)&\displaystyle \frac{\partial f}{\partial y}(A)&\displaystyle \frac{\partial f}{\partial z}(A) \end{array}\right)\left(\begin{array}{c}\displaystyle v_1\\ \displaystyle v_2\\ \displaystyle v_3\end{array}\right).$$
La différentielle peut \^etre aussi vue comme l'application qui à un vecteur associe son
produit scalaire par le vecteur des dérivées partielles au point $A$, qu'on appelle le gradient de $f$ au point $A$, et que l'on note
$$\nabla f(A)=\left(\frac{\partial f}{\partial x}(A),\frac{\partial f}{\partial y}(A),\frac{\partial f}{\partial z}(A)\right).$$

\vskip8mm

\section{Fonctions de classe \texorpdfstring{$\mathscr{C}^1$}%
{C1}}

\vskip4mm

\begin{definition}Soit $f:U\to \Rr$, o\`u $U$ est un ouvert de $\Rr^2$. On dit que $f$ est de classe ${\mathscr C}^1$ sur $U$ si les dérivées partielles $\displaystyle \frac{\partial f}{\partial x}$ et $\displaystyle \frac{\partial f}{\partial y}$ existent et sont continues sur $U$.
\end{definition}

\vskip4mm

\noindent{\bf Notation. }Soit $f:\Rr^2\to \Rr$ une fonction définie au voisinage de $(0,0)$. On dit que $f$ est négligeable par rapport à $\|(x,y)\|^n$ au voisinage de $(0,0)$ et on note $f=o\left(\|(x,y)\|^n\right)$ si 
$$\lim_{(0,0)}\frac{f(x,y)}{\|(x,y)\|^n}=0.$$

\vskip4mm

\begin{theoreme}Soit $f:U\to \Rr$ admettant des dérivées partielles premières sur l'ouvert $U$ qui soient continues au point $(a,b)\in U$. Alors pour tout $(h,k)\in \Rr^2$ tel que $(a+h,b+k)\in U$,
$$f(a+h,b+k)=f(a,b)+h\frac{\partial f}{\partial x}(a,b)+k\frac{\partial f}{\partial y}(a,b)+o\left(\|(h,k)\|\right).$$
On dit que $f$ admet un développement limité d'ordre $1$ au point $(a,b)$. En particulier, si $f$ est de classe ${\mathscr C}^1$ sur $U$ alors $f$ admet un développement limité d'ordre $1$ en tout point de $U$.
\end{theoreme}

\vskip4mm

\noindent{\it Démonstration. }Soit $(h,k)\in \Rr^2$ tel que $(a+h,b+k)\in U$. On a :
$$f(a+h,b+k)-f(a,b)=\left[f(a+h,b)-f(a,b)\right]+\left[f(a+h,b+k)-f(a+h,b)\right].$$
Les fonctions $x\mapsto f(x,b)$ et $y\mapsto f(a+h,y)$ sont dérivables respectivement aux points $a$ et $b$. Donc
$$f(a+h,b)-f(a,b)=h\frac{\partial f}{\partial x}(a,b)+o(h)$$
et
$$f(a+h,b+k)-f(a+h,b)=k\frac{\partial f}{\partial y}(a+h,b)+o(k).$$
Or, $\displaystyle \frac{\partial f}{\partial y}$ est continue au point $(a,b)$, donc $\displaystyle \frac{\partial f}{\partial y}(a+h,b)=\frac{\partial f}{\partial y}(a,b)+o(1)$. D'o\`u
$$f(a+h,b+k)-f(a,b)=h\frac{\partial f}{\partial x}(a,b)+k\frac{\partial f}{\partial x}(a,b)+h\varepsilon _1(h)+k\varepsilon _2(k)$$
avec $\displaystyle \lim _{h\to 0}\varepsilon _1(h)=0=\lim _{k\to 0}\varepsilon _2(k)$. Or
$$\frac{|h\varepsilon _1(h)+k\varepsilon _2(k)|}{\|(h,k)\|}\leq 2\left(|\varepsilon _1(h)|+|\varepsilon _2(k)|\right)\underset{(0,0)\; \; \; }{\longrightarrow 0}.$$
Donc
$$f(a+h,b+k)-f(a,b)=\d f_{(a,b)}(h,k)+o\left(\|(h,k)\|\right).$$
Ainsi $f$ admet un développement limité d'ordre $1$ au point $(a,b)$.

\vskip6mm

\noindent{\bf Remarques. }
\begin{enumerate}
\item On dit aussi que $f$ est différentiable au point $(a,b)$ si, et seulement si, $\d f_{(a,b)}$ existe et si
$$f(x,y)=f(a,b)+\d f_{(a,b)}(x-a,y-b)+o\left(\|(x-a,y-b)\|\right).$$
\item Comme en une seule variable, $f$ est différentiable au point $(a,b)$ si, et seulement si, $f$ est continue au point $(a,b)$ et y admet un développement limité d'ordre $1$.
\item Le théorème précédent implique que, toute fonction de classe $\mathscr{C}^1$ est différentiable. On retiendra que la réciproque est fausse.
\end{enumerate}

\vskip6mm

\noindent{\bf Exemple 1. }{\it Etudier la différentiabilité au point $(0,0)$ de la fonction $f$ définie par}
$$f(x,y)=\frac{x^4}{x^2+y^2}\;\mbox{ si }(x,y)\neq (0,0)\quad \mbox{et }\quad f(0,0)=0.$$
Il est clair que $f$ est continue au point $(0,0)$. Par ailleurs,
$$f(x,0)=x^2\; \rightarrow \; \lim _{x\to 0}\frac{f(x,0)-f(0,0)}{x}=\lim _{x\to 0}\frac{x^2}{x}=0$$
et 
$$f(0,y)=0\; \rightarrow \; \lim _{y\to 0}\frac{f(0,y)-f(0,0)}{y}=0.$$
Ainsi $\d f_{(0,0)}=0$ et, pour tout $(x,y)\neq (0,0)$, on a :
$$0\leq \frac{f(x,y)-f(0,0)-\d f_{(0,0)}(x,y)}{\sqrt{x^2+y^2}}=\frac{x^4}{(x^2+y^2)^{\frac{3}{2}}}\leq \frac{x^4}{|x|^3}=|x|\underset{(0,0)\; \; \; }{\longrightarrow 0}.$$
Donc $f$ est différentiable au point $(0,0)$.

\vskip6mm

\noindent{\bf Exemple 2. }{\it Soit $f:\Rr^2\to \Rr$ telle que
$$f(x,y)=y^2\sin \frac{1}{x^2+y^2}\; \; \mbox{ si }(x,y)\neq (0,0)\quad \mbox{ et }\quad f(0,0)=0.$$
Montrer que $f$ est différentiable en tout point de $\Rr^2$ sans \^etre de classe $\mathscr{C}^1$ sur $\Rr^2$.}
 
\vskip4mm

\noindent \underline{\it Sur $\Rr^2\setminus \{(0,0)\}$}. Les dérivées partielles 
$$\frac{\partial f}{\partial x}(x,y)=-\frac{2xy^2}{(x^2+y^2)^2}\cos \frac{1}{x^2+y^2}$$
et
$$\frac{\partial f}{\partial y}(x,y)=2y\sin \frac{1}{x^2+y^2}-\frac{2y^3}{(x^2+y^2)^2}\cos \frac{1}{x^2+y^2}$$
existent et sont continues sur $\Rr^2\setminus \{(0,0)\}$ : $f$ est de classe $\mathscr{C}^1$ sur $\Rr^2\setminus \{(0,0)\}$ et est donc différentiable sur $\Rr^2\setminus \{(0,0)\}$.

\vskip4mm

\noindent \underline{\it Différentiabilité au point $(0,0)$}. Calculons les dérivées partielles de $f$ au point $(0,0)$ :
$$f(x,0)=0 \rightarrow \lim _{x\to 0}\frac{f(x,0)-f(0,0)}{x}=0.$$
et
$$f(0,y)=y^2\sin \frac{1}{y^2} \rightarrow \lim _{y\to 0}\frac{f(0,y)-f(0,0)}{y}=\lim _{y\to 0}y\sin \frac{1}{y^2}=0$$
Donc $\d f_{(0,0)}$ existe et $\d f_{(0,0)}\equiv 0$. De plus
$$\lim _{(x,y)\to (0,0)}\frac{f(x,y)-f(0,0)-\d f_{(0,0)}(x,y)}{\sqrt{x^2+y^2}}=\lim _{(x,y)\to (0,0)}\frac{y^2}{\sqrt{x^2+y^2}}\sin \frac{1}{x^2+y^2}.$$
Or, $\displaystyle \left|\frac{y^2}{\sqrt{x^2+y^2}}\sin \frac{1}{x^2+y^2}\right|\leq |y|\underset{(0,0)\;\;}{\longrightarrow 0}$. Donc $f$ est différentiable au point $(0,0)$.

\vskip4mm

\noindent \underline{\it Conclusion}. La fonction $f$ est différentiable sur $\Rr^2$. Par ailleurs, 
$$\lim _{t\to 0}f'_x(t,t)=\lim _{t\to 0}\left[-\frac{1}{2t}\cos \frac{1}{2t^2}\right]$$ n'existe pas. Donc $f$ n'est pas de classe $\mathscr{C}^1$ sur $\Rr^2$.

\vskip6mm

\noindent La définition de $f$ de classe $\mathscr{C}^1$ sur $U$ ne suppose pas que $f$ est continue sur $U$. C'est une conséquence du résultat suivant :

\vskip6mm

\begin{corollaire}
Si la fonction $f:U\to \Rr$ est de classe $\mathscr{C}^1$ sur $U$, alors $f$ est continue en tout point de $U$.
\end{corollaire}

\vskip4mm

\noindent{\it Démonstration. }Soit $(a,b)\in U$. Puisque $f$ est de classe $\mathscr{C}^1$ sur $U$, on a :
$$f(a+h,b+k)=f(a,b)+\d f_{(a,b)}(h,k)+o\left(\|(h,k)\|\right).$$
Or, $\displaystyle \lim _{(0,0)}\d f_{(a,b)}(h,k)=0$ car $\d f_{(a,b)}$ est continue. Donc $\displaystyle \lim _{(0,0)}f(a+h,b+k)=f(a,b)$.

\vskip6mm

\noindent Plus intéressant, si $f$ est de classe $\mathscr{C}^1$ sur $U$, alors $f$ admet des dérivées suivant toutes les directions en tout point de $U$.

\vskip6mm

\begin{corollaire}Si $f:U\to \Rr$ est de classe $\mathscr{C}^1$ sur l'ouvert $U\subset \Rr^2$, alors, pour tout point $A\in U$ et tout vecteur non nul $\vec{v}=(v_1,v_2)$, $f$ admet une dérivée suivant le vecteur $\vec{v}$ au point $A$ et
$$D_{\vec{v}}f(A)=\d f_{A}(\vec{v})=\frac{\partial f}{\partial x}(A)v_1+\frac{\partial f}{\partial y}(A)v_2.$$
\end{corollaire}

\vskip4mm

\noindent{\it Démonstration. }Soit $(a,b)\in U$ et $\vec{v}=(v_1,v_2)\neq \vec{0}$. Avec le $DL_1$, on a :
$$f(a+tv_1,b+tv_2)=f(a,b)+t\d f_{(a,b)}(v_1,v_2)+o\left(\|(tv_1,tv_2)\|\right).$$
Donc
$$\frac{f(A+t\vec{v})-f(A)}{t}=\d f_{A}(v_1,v_2)+o(1)$$
car $o\left(\|(tv_1,tv_2)\|\right)=|t|\|(v_1,v_2)\|o(1)$. D'o\`u le résultat.

\vskip6mm

\noindent On retiendra que la réciproque de ce résultat est fausse.

\vskip6mm

\noindent{\bf Exemple. }{\it Soit $f:\Rr^2\to \Rr$ la fonction définie par
$$f(x,y)=\frac{y^3}{\sqrt{x^2+y^4}}\; \mbox{ si }(x,y)\neq (0,0)\quad \mbox{et }\quad f(0,0)=0.$$
Montrer que $f$ admet une dérivée suivant tout vecteur non nul au point $(0,0)$ mais qu'elle n'y est pas différentiable.}

\vskip4mm

\noindent \underline{\it Solution.}
\begin{enumerate}
\item Soit $\vec{v}=(v_1,v_2)\neq (0,0)$.
\begin{enumerate}
\item[.] Si $v_1=0$, on a $\displaystyle \frac{f(t\vec{v})-f(0,0)}{t}=\frac{f(0,tv_2)}{t}=v_2$.
\item[.] Si $v_1\neq 0$, on a $\displaystyle \left\vert \frac{f(t\vec{v})-f(0,0)}{t}\right\vert= \left\vert \frac{v_2^3t^2}{\sqrt{v_1^2t^2+v_2^4t^4}}\right\vert \leq \left\vert \frac{v_2^3}{v_1}\right\vert |t|\underset{t\to 0}{\longrightarrow 0}$.
\end{enumerate}
Ainsi $\displaystyle D_{\vec{v}}f(0,0)=\left\{\begin{array}{ccc}v_2&\mbox{si}&v_1=0\\ 0&\mbox{si}&v_1\neq 0. \end{array}\right.$
\item Avec $\vec{v}=(1,0)$, on aura $f'_x(0,0)=0$, avec $\vec{v}=(0,1)$, on aura $f'_y(0,0)=1$ et $f$ est continue au point $(0,0)$. Mais l'expression
$$\varepsilon (x,y)=\frac{f(x,y)-f(0,0)-\d f_{(0,0)}(x,y)}{\sqrt{x^2+y^2}}=\frac{y^3-y\sqrt{x^2+y^4}}{\sqrt{x^2+y^2}\sqrt{x^2+y^4}},$$
n'a pas de limite en $(0,0)$ car $\displaystyle \lim _{t\to 0^+}\varepsilon (0,t)=0$ et $\displaystyle \lim _{t\to 0^+}\varepsilon (t,t)=-\frac{1}{\sqrt{2}}$. Donc $f$ n'est pas différentiable au point $(0,0)$.
\end{enumerate}

\vskip6mm

\noindent{\bf Remarque. }La dérivée suivant le vecteur non nul $\vec{v}$ au point $A$ décrit la variation de $f$ autour de $A$ lorsqu'on se déplace dans la direction $\vec{v}$. La direction selon laquelle la croissance est la plus forte est celle du gradient de $f$. En effet,
$$D_{\vec{v}}f(A)=\langle \nabla f(A),\vec{v}\rangle=\| \nabla f(A)\|\, \| \vec{v}\|\, \cos \widehat{\left(\nabla f(A),\vec{v}\right)}.$$
Le maximum est atteint lorsque $\widehat{\left(\nabla f(A),\vec{v}\right)}=0$. C'est à dire lorsque $\displaystyle \vec{v}$ pointe dans la m\^eme direction que $\nabla f(A)$.

\vskip6mm

\begin{definition}Soit $F:U\subset \Rr^3\to \Rr$ de classe $\mathscr{C}^1$ sur $U$ et $S$ la surface définie par $S=\{(x,y,z)\in U\mid F(x,y,z)=0\}$. Soit $M_0\in S$ tel que $\nabla F(M_0)\neq \vec{0}$. Alors le plan tangent à $S$ au point $M_0$ est le plan passant par $M_0$ et de vecteur normal $\nabla F(M_0)$.
\end{definition}

\vskip6mm

\begin{proposition}Soit $f:U\subset \Rr^2\to \Rr$ de classe $\mathscr{C}^1$ sur $U$ et $(a,b)\in U$. Le plan tangent au graphe de $f$ existe en tout point du graphe et une équation cartésienne de ce plan au point $M_0=(a,b,f(a,b))$, est
$$\displaystyle z=f(a,b)+\frac{\partial f}{\partial x}(a,b).(x-a)+\frac{\partial f}{\partial y}(a,b).(y-b).$$
\end{proposition}

\vskip4mm

\noindent{\it Démonstration. }On introduit la fonction $F$ définie par
$$\forall (x,y,z)\in U\times \Rr,\;\; F(x,y,z)=z-f(x,y).$$
Le graphe de $f$ est la surface $\mathscr{G}_f=\{(x,y,z)\in U\times \Rr\mid F(x,y,z)=0\}$ et
$$\overrightarrow{\nabla}F(x,z,z)=(-f'_x(x,y),-f'_y(x,y),1)\neq \vec{0}.$$
Donc le plan tangent au graphe de $f$ existe en tout point du graphe.

\vskip8mm

\section{Différentielle d'une fonction composée}

\vskip4mm

\noindent Pour le calcul des différentielles, on est souvent amené à utiliser la règle de dérivation selon laquelle la différentielle d'une fonction composée est la composée des différentielles.

\vskip6mm

\begin{theoreme}Soit $f:U\subset \Rr^2\to \Rr$ et $u,v:I\subset \Rr\to \Rr$. On définit $\varphi :I\to \Rr^2$ par
$$\varphi (t)=\left(u(t),v(t)\right)$$
et on suppose que $f$ est de classe $\mathscr{C}^1$ sur $U$, $u$ et $v$ de classe $\mathscr{C}^1$ sur $I$ et $\varphi (I)\subset U$. Alors la fonction $F=f\circ \varphi $ est de classe $\mathscr{C}^1$ sur $I$ et
$$\forall t\in I,\; \; F'(t)=\frac{\partial f}{\partial x}(\varphi (t))u'(t)+\frac{\partial f}{\partial y}(\varphi (t))v'(t).$$
\end{theoreme}

\vskip4mm

\noindent{\it Démonstration. }Soit $t\in I$. En introduisant le vecteur dérivée $\varphi '(t)=(u'(t),v'(t))$, on a :
$$\varphi (t+h)=\varphi (t)+h\varphi '(t)+h\varepsilon (h)\quad \mbox{avec } \lim _{h\to 0}\varepsilon (h)=0.$$
Ici $\varepsilon$ est une fonction à valeurs dans $\Rr^2$. Or $f$ admet un $DL_1$ au point $\varphi (t)$ :
$$f\left(\varphi (t+h)\right)=f\left(\varphi (t)\right)+\d f_{\varphi (t)}\left(h\varphi '(t)+h\varepsilon (h)\right)+o\left(\|h\varphi '(t)+h\varepsilon (h)\|\right)$$
et comme $\|h\varphi '(t)+h\varepsilon (h)\|=|h|\|\varphi '(t)+\varepsilon (h)\|$, on voit que 
$$o\left(\|h\varphi '(t)+h\varepsilon (h)\|\right)=o(h).$$
En tenant compte de la linéarité de la différentielle, il en découle que
$$f\left(\varphi (t+h)\right)=f\left(\varphi (t)\right)+h\d f_{\varphi (t)}\left(\varphi '(t)\right)+h\d f_{\varphi (t)}\left(\varepsilon (h)\right)+o(h).$$
La forme linéaire $\d f_{\varphi (t)}$ étant continue, $\displaystyle \lim _{h\to 0}\d f_{\varphi (t)}\left(\varepsilon (h)\right)=0$. Donc
$$\lim _{h\to 0}\frac{f\left(\varphi (t+h)\right)-f\left(\varphi (t)\right)}{h}=\d f_{\varphi (t)}\left(\varphi '(t)\right).$$

\vskip6mm

\noindent{\bf Exemple. }{\it
\begin{enumerate}
\item Soit $f(x,y)=\e ^{x}\cos y$. Calculer la dérivée de la fonction $g:t\in \Rr\mapsto f(t^2,\sin t)$.
\item Soit $f:\Rr^2\to \Rr$ une fonction de classe $\mathscr{C}^1$. Soit $g:\Rr\to \Rr$ telle que
$$g(t)=f(2t,1+t^2).$$
Exprimer $g'(t)$ en fonction des dérivées partielles de $f$.
\end{enumerate}}

\vskip6mm

\noindent Une application $f:U\subset \Rr^n\to \Rr^m$ est dite de classe ${\mathscr C}^1$ sur $U$ si ses $m$ fonctions coordonnées le sont.

\vskip6mm

\begin{theoreme}Soit $f:U\subset \Rr^2\to \Rr$ de classe ${\mathscr C}^1$ sur $U$ et $g:V\subset \Rr^2\to \Rr^2$, définie par
$$\forall (u,v)\in V,\;\; g(u,v)=\left(g_1(u,v),g_2(u,v)\right),$$
de classe ${\mathscr C}^1$ sur $V$ telles que $g(V)\subset U$. Alors la fonction $F=f\circ g$ est de classe ${\mathscr C}^1$ sur $V$. Au point $(u,v)\in V$, sa différentielle est $\d F_{(u,v)}=\d f_{g(u,v)}\circ \d g_{(u,v)}$ et sa jacobienne est
$$\ja _{F}(u,v)=\ja _f(g(u,v))\times \ja _g(u,v).$$
C'est à dire,
$$\left\{\begin{array}{l}\displaystyle \frac{\partial F}{\partial u}(u,v)=\frac{\partial f}{\partial x}(g(u,v))\frac{\partial g_1}{\partial u}(u,v)+\frac{\partial f}{\partial y}(g(u,v))\frac{\partial g_2}{\partial u}(u,v)\\ \\ \displaystyle \frac{\partial F}{\partial v}(u,v)=\frac{\partial f}{\partial x}(g(u,v))\frac{\partial g_1}{\partial v}(u,v)+\frac{\partial f}{\partial y}(g(u,v))\frac{\partial g_2}{\partial v}(u,v).\end{array}\right.$$
\end{theoreme}

\vskip4mm

\noindent{\it Démontrastion. }Soit $(u,v)\in V$. On applique le théorème précédent à la première fonction partielle :
$$F_1:t\mapsto F(t,v)=f\left(g_1(t,v),g_2(t,v)\right).$$
Elle est dérivable au point $u$ et
$$\frac{\partial F}{\partial u}(u,v)=F_1'(u)=\frac{\partial f}{\partial x}(g(u,v))\frac{\partial g_1}{\partial u}(u,v)+\frac{\partial f}{\partial y}(g(u,v))\frac{\partial g_2}{\partial u}(u,v).$$
On fait de m\^eme avec la deuxième fonction partielle. On en déduit que $F$ admet des dérivées partielles premières en tout point de $V$. Ces dérivées s'expriment comme
composées de fonctions continues donc la fonction $F$ est de classe ${\mathscr C}^1$ sur $V$.

\vskip8mm

\section{Dérivées partielles d'ordre 2, hessienne}

\vskip4mm

\noindent Soit $f:(x,y,z)\mapsto f(x,y,z)$ une fonction de classe $\mathscr{C}^1$ sur l'ouvert $U$ de $\Rr^3$. Les tois dérivées partielles premières sont encore des fonctions de $U$ sur $\Rr$. Si elles-m\^emes sont de classe $\mathscr{C}^1$ sur $U$, on dit que $f$ est de classe $\mathscr{C}^2$ sur $U$. Leurs dérivées partielles, au nombre de $9$, sont les dérivées partielles secondes de $f$. Le théorème de Schwarz dit que le résultat ne dépend pas de l'ordre dans lequel on effectue les dérivations.

\vskip6mm

\begin{theoreme}[\bf Théorème de Schwarz]Soit $f:U\subset \Rr^n\to \Rr$ une fonction de classe ${\mathscr C}^2$. Pour tout $i,j=1,\dots ,n$, on a :
$$\frac{\partial}{\partial x_i}\left(\frac{\partial f}{\partial x_j}\right)=\frac{\partial}{\partial x_j}\left(\frac{\partial f}{\partial x_i}\right).$$
\end{theoreme}

\vskip4mm

\noindent La notation pour la dérivée partielle seconde par rapport à $x_i$ et $x_j$ est $\displaystyle \frac{\partial ^2f}{\partial x_i\partial x_j}$. Leur matrice est la matrice hessienne de $f$, qui est symétrique d'après le théorème de Schwarz.

\vskip6mm

\begin{definition}Soit $f:(x,y,z)\mapsto f(x,y,z)$ une fonction de classe $\mathscr{C}^2$ sur l'ouvert $U$ de $\Rr^3$. La matrice hessienne de $f$ au point $A\in U$ est
$$\mbox{H}_f(A)=\left(\begin{array}{ccc}\displaystyle \frac{\partial ^2f}{\partial x^2}(A)&\displaystyle \frac{\partial ^2f}{\partial y\partial x}(A)&\displaystyle \frac{\partial ^2f}{\partial z\partial x}(A)\\ \\ \displaystyle \frac{\partial ^2f}{\partial x\partial y}(A)&\displaystyle \frac{\partial ^2f}{\partial y^2}(A)&\displaystyle \frac{\partial ^2f}{\partial z\partial y}(A)\\ \\ \displaystyle \frac{\partial ^2f}{\partial z\partial x}(A)&\displaystyle \frac{\partial ^2f}{\partial z\partial y}(A)&\displaystyle \frac{\partial ^2f}{\partial z^2}(A)\end{array}\right).$$
\end{definition}

\vskip6mm

\noindent{\bf Exemple. }Soit $f$ la fonction définie par
$$f(x,y)=\frac{xy^3}{x^2+y^2}\;\mbox{ si }(x,y)\neq (0,0)\quad \mbox{et}\quad f(0,0)= 0.$$
On vérifie que $f$ est de classe $\mathscr{C}^1$ sur $\Rr^2$ et que
$$\frac{\partial f}{\partial x}(x,y)=\left\{\begin{array}{cl}\displaystyle \frac{y^5-x^2y^3}{(x^2+y^2)^2}&\mbox{si }(x,y)\neq (0,0)\\ 0&\mbox{si }(x,y)=(0,0)
\end{array}\right.$$
et 
$$\frac{\partial f}{\partial y}(x,y)=\left\{\begin{array}{cl}\displaystyle \frac{3x^3y^2+xy^4}{(x^2+y^2)^2}&\mbox{si }(x,y)\neq (0,0)\\ 0&\mbox{si }(x,y)=(0,0).\end{array}\right.$$
Le taux d'accroissement
$$\frac{\frac{\partial f}{\partial x}(0,y)-\frac{\partial f}{\partial x}(0,0)}{y-0}=1\underset{y\to 0\; \; \; }{\longrightarrow 1}$$
ce qui montre que $\displaystyle \frac{\partial ^2f}{\partial y\partial x}(0,0)=1$.
De m\^eme, le taux d'accroissement
$$\frac{\frac{\partial f}{\partial y}(x,0)-\frac{\partial f}{\partial y}(0,0)}{x-0}=0\underset{x\to 0\; \; \; }{\longrightarrow 0}$$
ce qui montre que $\displaystyle \frac{\partial ^2f}{\partial x\partial y}(0,0)=0$. On en déduit que l'une des dérivées partielles secondes $\displaystyle \frac{\partial ^2f}{\partial x\partial y}$ ou $\displaystyle \frac{\partial ^2f}{\partial y\partial x}$ n'est pas continue en $(0,0)$.

\vskip6mm

\begin{theoreme}[\bf Formule de Taylor]Soit $f:U\to \Rr$ une fonction de classe ${\mathscr C}^2$ sur l'ouvert $U\subset \Rr^2$ et soit $(a,b)\in U$. Alors, pour tout $(x,y)\in U$,
$$\begin{array}{ccl}f(x,y)&=&\displaystyle f(a,b)+(x-a)\frac{\partial f}{\partial x}(a,b)+(y-b)\frac{\partial f}{\partial y}(a,b)\\ \\ & &\displaystyle +\frac{1}{2}\left[(x-a)^2\frac{\partial ^2f}{\partial x^2}(a,b)+2(x-a)(y-b)\frac{\partial ^2f}{\partial x\partial y}(a,b)+(y-b)^2\frac{\partial ^2f}{\partial y^2}(a,b)\right]\\ \\ & &\displaystyle +o\left(\|(x-a,y-b)\|^2\right).\end{array}$$
On dit aussi que $f$ admet un développement limité d'ordre $2$ au point $(a,b)$.
\end{theoreme}

\vskip8mm

\section{Extremum et points critiques}

\vskip4mm

\begin{definition}Soit $f:U\subset \Rr^2\to \Rr$, o\`u $U$ est un ouvert. On dit que $f$ admet un maximum (resp. minimum) local en $A\in U$, s'il existe une boule ouverte $B\subset U$, centrée en $A$, telle que :
$$\forall M\in B,\; f(M) \leq f(A),\: (\mbox{resp. }f(M)\geq f(A).$$
On dit que $f$ admet un extremum local en $A$, si elle y admet un maximum local ou un minimum local.
\end{definition}

\vskip4mm

\begin{proposition}Soit $f:U\subset \Rr^2\to \Rr$ une fonction de classe $\mathscr{C}^1$ sur un ouvert $U$ et $A\in U$. Si $f$ possède un extremum local en un point $A$, alors le gradient de $f$ au point $A$ est nul : $\displaystyle \frac{\partial f}{\partial x}(A)=\frac{\partial f}{\partial y}(A)=0$.
\end{proposition}

\vskip2mm

\noindent{\it Démonstration. }Soit $\vec{v}\in \Rr^2$. La fonction $\varphi :t\mapsto f(A+t\vec{v})$, définie pour $t$ assez petit, est dérivable au voisinage de $0$ et admet un extremum en $0$. Donc $\varphi '(0)=0$ et puis
$$0=\varphi '(0)=\d _Af(\vec{v})=\langle \nabla f(A),\vec{v}\rangle.$$
Or ceci est vrai pour tout vecteur $\vec{v}$, donc $\nabla f(A)=0$.

\vskip6mm

\noindent Les points de $U$ o\`u le gradient de $f$ s'annule sont appelés points critiques de $f$. Le résultat précédent dit que les extremums d'une fonction ne peuvent se produire qu'en un point critique. La réciproque est fausse.

\vskip6mm

\begin{theoreme}Soit $U$ un ouvert de $\Rr^2$, $f$ une fonction de classe $\mathscr{C}^2$ sur $U$ et $(a,b)\in U$ un point critique de $f$.
\begin{enumerate}
\item Si $\mathrm{H}_f(a,b)$ a toutes ses valeurs propres strictement positives, alors $f$ présente un minimum en $(a,b)$.
\item Si $\mathrm{H}_f(a,b)$ a toutes ses valeurs propres strictement négatives, alors $f$ présente un maximum en $(a,b)$.
\item Si $\mathrm{H}_f(a,b)$ a deux valeurs propres de signes opposés, alors $f$ ne présente pas d'extremum en $(a,b)$. On dit que $(a,b)$ est un point selle.
\item Dans les autres cas, on ne peut rien dire (tout peut arriver).
\end{enumerate}
\end{theoreme}

\vskip4mm

\noindent{\it Démonstration. }Pour $h$ et $k$, assez petits, non nuls, la différence $f(a+h,b+k)-f(a,b)$ est du signe de
$$q(h,k)=\frac{\partial ^2f}{\partial x^2}(a,b)h^2+2\frac{\partial ^2f}{\partial x\partial y }(a,b)hk+\frac{\partial ^2f}{\partial y^2}(a,b)k^2=\left(\begin{array}{cc}h&k
\end{array}\right)\mbox{H}_f(a,b)\left(\begin{array}{c}h\\k\end{array}\right).$$
Il se trouve que, comme pour toute matrice symétrique réelle, $\mbox{H}_f(a,b)$ est diagonalisable : il existe une matrice orthogonale $P\in \mathscr{M}_2(\Rr)$, $P^{-1}={^tP}$, et deux réels $\lambda$ et $\mu$ tels que 
$$\mbox{H}_f(a,b)=P\left(\begin{array}{cc}\lambda &0\\ 0&\mu\end{array}\right)P^{-1}.$$
Posons
$$\left(\begin{array}{c}h^*\\k^*\end{array}\right)={^tP}\left(\begin{array}{c}h\\k\end{array}\right) \Leftrightarrow \left(\begin{array}{cc}h^*&k^*\end{array}\right)=\left(\begin{array}{cc}h&k\end{array}\right)P.$$
La quantité $q(h,k)$ s'écrit
$$q(h,k)=\left(\begin{array}{cc}h^*&k^*\end{array}\right)\left(\begin{array}{cc}\lambda &0\\ 0&\mu\end{array}\right)\left(\begin{array}{c}h^*\\k^*\end{array}\right)=\lambda (h^*)^2+\mu (k^*)^2.$$
Le signe de $q(h,k)$ dépend donc du signe des valeurs propres $\lambda$ et $\mu$.
\begin{enumerate}
\item[$\bullet$] Si $\lambda >0$ et $\mu >0$, alors $q(h,k)>0$. On a un minimum local.
\item[$\bullet$] Si $\lambda <0$ et $\mu <0$, alors $q(h,k)<0$. On a un maximum local.
\item[$\bullet$] Si $\lambda >0$ et $\mu <0$, alors $q(h,k)>0$ dans la direction $(h^*,0)P$, et $q(h,k)<0$ dans la direction $(0,k^*)P$. On a un point selle.
\end{enumerate}

\vskip6mm

\noindent{\bf Rappel. }Les valeurs propres de $\mbox{H}_f(a,b)$ sont les racines du polyn\^ome caractéristique de $\mbox{H}_f(a,b)$ qui est défini par $\chi (X)=\det \left[\mbox{H}_f(a,b)-XI_2\right]$.

\vskip6mm

\noindent{\bf Exemple 1. }Soit $f(x,y)=x^2+y^2$. Le point $(0,0$) est l'unique point critique de $f$ et
$$\mbox{H}_f(0,0)=2I_2.$$
Les valeurs propres de la hessienne de $f$ au point $(0,0)$ sont toutes les deux strictement positives. Donc $f$ admet un minimum local en $(0,0)$.

\vskip6mm

\noindent{\bf Exemple 2. }Soit $f(x,y)=x^2-y^2$. On trouve un seul point critique : $(0,0)$. La hessienne de $f$ au point $(0,0)$, $\mbox{H}_f(0,0)=\left(\begin{array}{cc}2 &0\\ 0&-2\end{array}\right)$, admet une valeur propre strictement positive et une valeur propre strictement négative. Donc $(0,0)$ n'est ni un maximum ni un minimum ; c'est un point selle (cf. Figure 1 : selle de cheval).

\vskip6mm

\noindent{\bf Exemple 3. }Soit $f(x,y)=x^3-3xy^2$. On trouve un seul point critique : $(0,0)$. La hessienne de $f$ au point $(0,0)$, $\mbox{H}_f(0,0)=(0)$, a deux valeur propre nulles. On ne peut pas conclure. Or, $f_0:x\mapsto f(x,0)=x^3$ possède un point d'inflexion au point $0$. Donc $(0,0)$ est un point selle (cf. Figure 2 : selle de singe).

\vskip6mm

\noindent{\bf Exemple 4. }Soit $f(x,y)=x^4+y^4-2x^2$. On trouve trois points critiques $(-1,0)$, $(1,0)$ et $(0,0)$. Par ailleurs,
$$\mbox{H}_f(-1,0)=\mbox{H}_f(1,0)=\left(\begin{array}{cc}8&0\\ 0&0\end{array}\right)\quad \mbox{ et }\quad \mbox{H}_f(0,0)=\left(\begin{array}{cc}-4&0\\ 0&0\end{array}\right).$$
On ne peut pas conclure. Or
$$f(x,y)=(x^2-1)^2+y^4-1\geq -1=f(\pm 1,0).$$
Donc $f$ admet un minimum aux points $(\pm 1,0)$. D'autre part, pour $|x|\leq 1$,
$$f(x,0)=x^4-2x^2=x^2(x^2-2)\leq 0=f(0,0)\quad \mbox{ et }\quad f(0,y)=y^4\geq 0=f(0,0).$$
Donc $(0,0)$ n'est ni un maximum ni un minimum ; c'est un point selle.

\vskip6mm

\noindent{\bf Exemple 5. }Soit $f(x,y)=2x^3-y^4-3x^2$. On trouve deux points critiques $(0,0)$ et $(1,0)$. Par ailleurs,
$$\mbox{H}_f(1,0)=\left(\begin{array}{cc}6&0\\ 0&0\end{array}\right)\quad \mbox{ et }\quad \mbox{H}_f(0,0)=\left(\begin{array}{cc}-6&0\\ 0&0\end{array}\right).$$
On ne peut pas conclure. Or, pour $|x-1|\leq 1$,
$$f(1,y)=-1-y^4\leq -1=f(1,0)\quad \mbox{ et }\quad f(x,0)=(x-1)^2(2x+1)-1\geq -1=f(1,0).$$
Donc $(1,0)$ n'est ni un maximum ni un minimum ; c'est un point selle. D'autre part, pour $|x|\leq 1$,
$$f(x,y)=x^2(2x-3)-y^4\leq 0=f(0,0).$$
Donc $f$ admet un maximum local au point $(0,0)$.

%\newpage
%
%\vskip6mm
%
%\begin{center}
%\begin{pspicture}(-7,-5.8)(6,4.5)
%\psset{unit=0.6}
%\psset{viewpoint=40 55 25 rtp2xyz,Decran=50}
%\psset{lightsrc=viewpoint}
%\psSurface[ngrid=0.15 0.15,incolor=yellow,linewidth=0.4
%\pslinewidth,algebraic,hue=0 1](-4,-4)(4,4){%
%0.25*((x^2)-(y^2))}
%\axesIIID[linewidth=1pt,linecolor=black,labelsep=6pt](1.5,0,0)(5.5,7.5,4.5)
%\psSolid[object=parallelepiped,a=8,b=8,c=8,action=draw,linecolor=black]
%\quadrillage
%\Rrput(0,-9){Figure 1 : selle de cheval}
%\end{pspicture}
%\end{center}
%
%\vskip8mm
%
%\begin{center}
%\begin{pspicture}(-7,-6.8)(6,6)
%\psset{unit=0.6}
%\psset{viewpoint=50 60 22 rtp2xyz,Decran=60}
%\psset{lightsrc=viewpoint}
%\psSurface[ngrid=0.15 0.15,incolor=yellow,linewidth=0.4
%\pslinewidth,algebraic,hue=0 1](-4,-4)(4,4){%
%0.05*((x^3)-3*x*(y^2))}
%\axesIIID[linewidth=1pt,linecolor=black,labelsep=6pt](3,0,0)(7,7,7.5)
%\psSolid[object=parallelepiped,a=8,b=8,c=12.8,action=draw,linecolor=black]
%\quadrillage
%\Rrput(0,-11){Figure 2 : selle de singe}
%\end{pspicture}
%\end{center}


\chapter{\bf Difféomorphismes, EDP}
\section{Difféomorphismes}

\thispagestyle{empty}

\vskip4mm

\noindent Les applications de $\Rr^n$ dans $\Rr^n$ qui sont bijectives et de classe $\mathscr{C}^1$ ainsi que leur réciproque, sont utilisées comme changements de variables. On les appelle des difféomorphismes.

\vskip6mm

\begin{definition}Soient $U$ et $V$ deux ouverts de $\Rr^n$ et $\Phi:U\to V$. On dit que $f$ est un ${\mathscr C}^1$-difféomorphisme si
\begin{enumerate}
\item $\Phi$ est une bijection de $U$ sur $V$.
\item $\Phi$ est de classe ${\mathscr C}^1$ sur $U$.
\item $\Phi^{-1}$ est de classe ${\mathscr C}^1$ sur $V$.
\end{enumerate}
\end{definition}

\vskip4mm

\noindent Du théorème de composition découle que les différentielles de $\Phi$ et $\Phi^{-1}$ sont elles aussi réciproques l'une de l'autre. Et donc les matrices jacobiennes, qui sont des matrices carrées $n\times n$, sont inverses l'une de l'autre.

\vskip6mm

\begin{proposition}Soit $\Phi:U\to V$ un ${\mathscr C}^1$-difféomorphisme, $A\in U$ et $B\in V$. Alors
$$\left[\ja _{\Phi}(A)\right]^{-1}=\ja _{\Phi ^{-1}}(\Phi(A))\quad \mbox{et}\quad \ja _{\Phi ^{-1}}(B)=\left[\ja _{\Phi}\left(\Phi^{-1}(B)\right)\right]^{-1}.$$
\end{proposition}

\vskip4mm

\noindent Pour un difféomorphisme, le déterminant de la matrice jacobienne joue un r\^ole particulier.

\vskip6mm

\begin{definition}Soient $U$ et $V$ deux ouverts de $\Rr^n$ et $\Phi:U\to V$ une application de classe ${\mathscr C}^1$. On appelle jacobien de $\Phi$ au point $A\in U$ le déterminant de la matrice jacobienne de $\Phi$ au point $A$ :
$$\jac _{\Phi}(A)=\det \left(\ja _{\Phi}(A)\right).$$
\end{definition}

\vskip4mm

\noindent Il est clair que le jacobien d'un difféomorphisme ne s'annule pas, puisque la matrice jacobienne est inversible. La réciproque est donnée par le théorème d'inversion.

\vskip6mm

\begin{theoreme}[d'inversion]Soient $U$ et $V$ deux ouverts de $\Rr^n$ et $\Phi:U\to V$ une application de classe ${\mathscr C}^1$. Si $\Phi$ est bijective et si le jacobien de $\Phi$ ne s'annule pas sur $U$, alors $\Phi$ est un ${\mathscr C}^1$-difféomorphisme de $U$ sur $V$.
\end{theoreme}

\vskip4mm

\noindent{\bf Exemple. }Les passages en coordonnées polaires, cylindriques ou sphériques, sont très souvent utilisés. Détaillons le premier qui consiste à remplacer les coordonnées cartésiennes $(x,y)$ d'un point du plan, par le module $r$ et l'argument $\theta$ du point dans le plan complexe.
$$\begin{array}{ccccl}\Phi &:&U=\Rr^2\setminus \left(\Rr^+\times\{0\}\right)&\to&V=]0,+\infty [\times ]0,2\pi [\\&&(x,y)&\mapsto &\displaystyle (r,\theta).\end{array}$$

\vskip4mm

\noindent Dans la pratique, on travaille avec la réciproque
$$\begin{array}{ccccl}\Psi &:&V&\to&U\\&&(r,\theta)&\mapsto &\displaystyle (x,y)\end{array} \qquad \mbox{ o\`u }\left\{\begin{array}{l} x=r \cos \theta \\ y=r \sin \theta .\end{array}\right.$$
On doit avoir $r=\sqrt{x^2+y^2}$ et le point $(x/r,y/r)$ est dans le cercle unité privé du point $(1,0)$. Donc il existe un unique $\theta \in ]0,2\pi [$ tel que
$$x=r\cos \theta \quad \mbox{et}\quad y=r\sin \theta .$$
Ainsi $\Psi$ est bijective et il est évident qu'elle est de classe ${\mathscr C}^1$. Sa matrice jacobienne est 
$$\ja _{\Psi }(r,\theta )=\left(\begin{array}{cc}\cos \theta &-r\sin \theta \\\displaystyle \sin \theta &r\cos \theta \end{array}\right)$$
et son jacobien, qui vaut $r$, ne s'annule pas sur $V$. Donc $\Psi $ est un ${\mathscr C}^1$-difféomorphisme de $V$ sur $U$. Pour calculer les dérivées partielles de $r$ et $\theta$, on utilise l'inversion matricielle de la jacobienne. En effet, puisque $\Phi =\Psi ^{-1}$,
$$\ja _{\Phi }(x,y)=\left(\begin{array}{cc}\displaystyle \frac{\partial r}{\partial x}&\displaystyle \frac{\partial r}{\partial y}\\ \displaystyle \frac{\partial \theta}{\partial x}&\displaystyle \frac{\partial \theta}{\partial y}\end{array}\right)=\left[\ja _{\Psi }(r,\theta ))\right]^{-1}=\frac{1}{r}\left(\begin{array}{cc}r\cos \theta &r\sin \theta \\ \\ -\sin \theta &\cos \theta\end{array}\right)$$
Ce qui nous donne
$$\left(\begin{array}{cc}\displaystyle \frac{\partial r}{\partial x}&\displaystyle \frac{\partial r}{\partial y}\\ \\ \displaystyle \frac{\partial \theta}{\partial x}&\displaystyle \frac{\partial \theta}{\partial y}\end{array}\right)=\left(\begin{array}{cc}\cos \theta &\sin \theta \\ \\ \displaystyle -\frac{\sin \theta }{r}&\displaystyle \frac{\cos \theta }{r}\end{array}\right)=\left(\begin{array}{cc}\displaystyle \frac{x}{\sqrt{x^2+y^2}} &\displaystyle \frac{y}{\sqrt{x^2+y^2}} \\ \\ \displaystyle -\frac{y}{x^2+y^2}&\displaystyle \frac{x}{x^2+y^2}\end{array}\right).\leqno{(*)}$$

\vskip4mm

\noindent Considérons maintenant une application $f:(x, y)\mapsto f(x,y)$ de $U$ dans $\Rr$. Pour utiliser passer en coordonnées, on doit remplacer les anciennes coordonnées $(x,y)$ par les nouvelles coordonnées $(r,\theta)$, et donc considérer la fonction $g$ de $V$ dans $\Rr$ qui à $(r,\theta)$ associe :
$$g(r,\theta)=f\left(\Phi ^{-1}(r,\theta)\right)=f\left(x(r,\theta),y(r,\theta)\right).$$
La formule de dérivation des fonctions composées donne
$$\left\{\begin{array}{ccl}\displaystyle \frac{\partial f}{\partial x}(x,y)&=&\displaystyle \frac{\partial g}{\partial r}(r,\theta )\frac{\partial r}{\partial x}(x,y)+\frac{\partial g}{\partial \theta}(r,\theta )\frac{\partial \theta }{\partial x}(x,y)\\ \\ \displaystyle \frac{\partial f}{\partial y}(x,y)&=&\displaystyle \frac{\partial g}{\partial r}(r,\theta )\frac{\partial r}{\partial y}(x,y)+\frac{\partial g}{\partial \theta}(r,\theta )\frac{\partial \theta }{\partial x}(x,y).\end{array}\right.$$
Donc, d'après $(*)$, on aura :
$$\left\{\begin{array}{ccl}\displaystyle \frac{\partial f}{\partial x}(x,y)&=&\displaystyle \cos \theta .\frac{\partial g}{\partial r}(r,\theta )-\frac{\sin \theta }{r}.\frac{\partial g}{\partial \theta}(r,\theta )\\ \\ \displaystyle \frac{\partial f}{\partial y}(x,y)&=&\displaystyle \sin \theta .\frac{\partial g}{\partial r}(r,\theta )+\frac{\cos \theta }{r}.\frac{\partial g}{\partial \theta}(r,\theta ).\end{array}\right. \leqno{(\star \star )}$$

\vskip8mm

\section{Equations aux dérivées partielles}
\subsection{Champs de vecteurs}

\vskip4mm

\noindent Les équations aux dérivées partielles sont omniprésentes en physique. Elles relient entre elles les dérivées partielles d'ordre $1$ et $2$, et font intervenir des combinaisons de dérivées partielles comme le gradient, la divergence ou le rotationnel.

\vskip6mm

\noindent On rappelle que le gradient d'une fonction de deux variables $f$ est le champ de vecteurs de $\Rr^2$ défini par
$$\nabla f=\left(\frac{\partial f}{\partial x},\frac{\partial f}{\partial y}\right).$$
On dispose donc d'un opérateur, noté formellement, $\displaystyle \nabla :=\left(\frac{\partial }{\partial x},\frac{\partial }{\partial y}\right)$ sur les fonctions. De m\^eme, le gradient d'une fonction de trois variables $f$ est le champ de vecteurs de $\Rr^3$ défini par
$$\nabla f=\left(\frac{\partial f}{\partial x},\frac{\partial f}{\partial y},\frac{\partial f}{\partial z}\right).$$
On dispose à nouveau d'un opérateur, noté formellement, $\displaystyle \nabla :=\left(\frac{\partial }{\partial x},\frac{\partial }{\partial y},\frac{\partial }{\partial z}\right)$.

\vskip6mm

\begin{definition}Soit $U$ un ouvert de $\Rr^2$. Soit $F:(x,y)\mapsto \left(P(x,y),Q(x,y)\right)$ une application de classe $\mathscr{C}^1$ de $U$ dans $\Rr^2$. Une telle application est aussi appelée un champ de vecteurs de $\Rr^2$ défini sur $U$. On définit formellement le rotationnel du champ de vecteurs $F$ comme étant le champ de vecteurs de $\Rr$ défini sur $U$ par
$$\mathrm{rot}(F)(x,y)=\det (\nabla ,F)=\left|\begin{array}{cc}\frac{\partial}{\partial x}&P\\ \\ \frac{\partial}{\partial y}&Q
\end{array}\right|(x,y)=\frac{\partial Q}{\partial x}(x,y)-\frac{\partial P}{\partial y}(x,y).$$
\end{definition}

\vskip4mm

\noindent Un champ de vecteurs sera noté indifféremment $F$ ou $\overrightarrow{F}$. On vérifiera à partir de cette définiton et le théorème de Schwarz que, $\mbox{rot}(\nabla f)=0$.

\vskip6mm

\begin{definition}Soit $U$ un ouvert de $\Rr^3$ et $F:(x,y,z)\mapsto \left(P(x,y,z),Q(x,y,z),R(x,y,z)\right)$ une application de classe $\mathscr{C}^1$ de $U$ dans $\Rr^3$, appelée aussi champ de vecteurs de $\Rr^3$ défini sur $U$.
\begin{enumerate}
\item Le rotationnel de $F$ est le champ de vecteurs de $\Rr^3$ donné par
$$\mathrm{rot}(F)=\nabla \wedge F=\left(\frac{\partial R}{\partial y}-\frac{\partial Q}{\partial z},\frac{\partial P}{\partial z}-\frac{\partial R}{\partial x},\frac{\partial Q}{\partial x}-\frac{\partial P}{\partial y}\right).$$
\item La divergence de $F$ est la fonction $\displaystyle \mathrm{div}(F)=\langle \nabla ,F\rangle =\frac{\partial P}{\partial x}+\frac{\partial Q}{\partial y}+\frac{\partial R}{\partial z}$.
\end{enumerate}
\end{definition}

\vskip4mm

\noindent On vérifiera à partir de ces définitons et le théorème de Schwarz que, $\mbox{rot}(\nabla f)=0$ et que, pour un champ de vecteurs $F$ de $\Rr^3$, $\mathrm{div}(\mathrm{rot}(F))=0$.

\vskip6mm

\begin{definition}Soit $F$ un champ de vecteurs défini sur $U$. On dit que $F$ dérivé d'un potentiel sur $U$ s'il existe une fonction $f:U\to \Rr$ telle que $F= \nabla f$ sur $U$. Dans ce cas, on dira que $f$ est un potentiel de $F$.
\end{definition}

\vskip4mm

\begin{theoreme}[\bf Poincaré]Soit $U$ un ouvert simplement connexe de $\Rr^2$ (resp. $\Rr^3$) et $F$ un champ de vecteurs de $\Rr^2$ (resp. $\Rr^3$) de classe ${\mathscr C}^1$ sur $U$. Alors $F$ dérive d'un potentiel sur $U$ si, et seulement si, $\Rrot F=0$.
\end{theoreme}

\vskip6mm

\noindent{\bf Méthode. }Lorsqu'un champ de vecteurs $\overrightarrow{F}$ dérive d'un potentiel $f$, on écrit $\displaystyle \nabla f=\overrightarrow{F}$. En identifiant les coordonnées, on obtient un système d'équations dont la seule inconnue est $f$. Il faut donc intégrer ce système pour déterminer $f$.

\vskip6mm

\noindent{\bf Exemple. }{\it Montrer que le champ de vecteurs $\overrightarrow{F}(x,y)=y^2\vec{i}+(2xy-1)\vec{j}$ dérive d'un potentiel sur $\Rr^2$ et déterminer les potentiels dont il dérive.}

\vskip4mm

\noindent \underline{\it Solution}. \rm Ici $P(x,y)=y^2$, $Q(x,y)=2xy-1$ et $\displaystyle \frac{\partial P}{\partial y}=2y=\frac{\partial Q}{\partial x}$. Donc $\Rrot \overrightarrow{F}=0$ et, comme $\Rr^2$ est simplement connexe, $\overrightarrow{F}$ dérive d'un potentiel $f$ sur $\Rr^2$. On aura :
$$\frac{\partial f}{\partial x}(x,y)=P(x,y)=y^2\rightarrow f(x,y)=xy^2+K(y)$$
et 
$$\frac{\partial f}{\partial y}(x,y)=Q(x,y)=2xy-1\rightarrow K'(y)=-1\rightarrow K(y)=-y+C,\quad C\in \Rr.$$
Les potentiels de $\overrightarrow{F}$ sur $\Rr^2$ sont les fonctions $f$ définies par $f(x,y)=xy^2-y+C$.

\vskip8mm

\subsection{Exemples d'équations aux dérivées partielles}

\vskip4mm

\noindent Soit $U$ un ouvert non vide de $\Rr^2$. On note $(x_0,y_0)$ un point de $U$ et $U_1$ (resp. $U_2$) la projection de $U$ sur l'axe $y=0$ (resp. $x=0$).

\vskip6mm

\begin{proposition}Soit $h$ une fonction de classe $\mathscr{C}^0$ sur $U$. On note $H$ la primitive de $h_1:x\mapsto h(x,y)$ sur $U_1$ qui s'annule en $x_0$. Une fonction $f$ de classe $\mathscr{C}^1$ sur $U$ est une solution de 
$$(E_1)\; :\; \frac{\partial f}{\partial x}(x,y)=h(x,y)$$
si, et seulement si, il existe une fonction $k$ de classe $\mathscr{C}^1$ sur $U_2$ telle que
$$\forall (x,y)\in U,\; \; f(x,y)=H(x,y)+k(y).$$
\end{proposition}

\vskip4mm

\noindent{\it Démonstration. }Si $f$ est une solution de $(E_1)$ la fonction $\varphi :x\mapsto f(x,y)-H(x,y)$ est dérivable et de dérivée nulle. Elle est donc constante :
$$\forall x\in U_1,\; \varphi (x)=\varphi (x_0)\rightarrow f(x,y)=H(x,y)+f(x_0,y)$$
et $k:y\mapsto f(x_0,y)$ est bien une fonction de classe $\mathscr{C}^1$ sur $U_2$. Réciproquement, on vérifie qu'une fonction de cette forme est solution de $(E_1)$.

\vskip6mm

\begin{proposition}Soit $h$ une fonction de classe $\mathscr{C}^0$ sur $U_1$ et $H$ une primitive de $h$ sur $U_1$. Une fonction $f$ de classe $\mathscr{C}^2$ sur $U$ est une solution de 
$$(E_2)\; :\; \frac{\partial ^2f}{\partial x\partial y}(x,y)=h(x)$$
si, et seulement si, il existe une fonction $K$ de classe $\mathscr{C}^2$ sur $U_2$ telle que
$$\forall (x,y)\in U,\; \; f(x,y)=yH(x)+K(y).$$
\end{proposition}

\vskip4mm

\noindent{\it Démonstration. }Si $f$ est une solution de $(E_2)$ la fonction $\displaystyle \frac{\partial f}{\partial y}$ est solution d'une équation du type $(E_1)$. Donc
$$\forall (x,y)\in U,\; \frac{\partial f}{\partial y}(x,y)=H(x)+k(y)$$
o\`u $k$ est une fonction de classe $\mathscr{C}^1$ sur $U_2$. Ainsi $f$ est une solution d'une équation du type $(E_1)$. Donc de la forme ci-dessus.  Réciproquement, on vérifie qu'une fonction de cette forme est solution de $(E_2)$.

\vskip6mm

\begin{proposition}Une fonction $f$ de classe $\mathscr{C}^2$ sur $U$ est une solution de 
$$(E_3)\; :\; \frac{\partial ^2f}{\partial x^2}(x,y)=0$$
si, et seulement si, il existe deux fonctions $K$ et $H$ de classe $\mathscr{C}^2$ sur $U_2$ telles que
$$\forall (x,y)\in U,\; \; f(x,y)=xH(y)+K(y).$$
\end{proposition}

\vskip4mm

\noindent{\it Démonstration. }Si $f$ est une solution de $(E_3)$ la fonction $\displaystyle \frac{\partial f}{\partial x}$ est solution d'une équation du type $(E_1)$. Donc
$$\forall (x,y)\in U,\; \frac{\partial f}{\partial x}(x,y)=k(y)$$
o\`u $k$ est une fonction de classe $\mathscr{C}^1$ sur $U_2$. Ainsi $f$ est une solution d'une équation du type $(E_1)$. Donc de la forme ci-dessus. Réciproquement, on vérifie qu'une fonction de cette forme est solution de $(E_3)$.

\vskip6mm

\noindent{\bf Résolution à l'aide d'un difféomorphisme. }Pour intégrer une EDP, $(E)$ donnée, on utilise un changement de variables pour se ramener à une EDP plus simple. Soit
$$\begin{array}{ccccl}\Phi &:&U&\to&V\\&&(x,y)&\mapsto &\displaystyle (u,v).\end{array}$$
un $\mathscr{C}^1$-difféomorphisme. Pour une fonction $f$ solution de $(E)$, on pose $g=f\circ \Phi ^{-1}$. C'est à dire $f=g\circ \Phi$.
\begin{enumerate}
\item On utilise la formule de dérivation des fonctions composées pour exprimer les dérivées partielles de $f$ en fonction de $g$, $u$ et $u$.
\item On remplace dans l'équation $(E)$ ce qui donne l'EDP $(E')$ satisfaite par $g$.
\item On intègre $(E')$ et on en déduit les solutions $f$ de $(E)$.
\end{enumerate}

\vskip6mm

\noindent{\bf Exemple. }Intégrons dans $U=\{(x,y)\in \Rr^2|\ x>0\}$ l'EDP suivante :
$$(E)\;\; :\; \; x\frac{\partial f}{\partial x}+y\frac{\partial f}{\partial y}=\sqrt{x^2+y^2}.$$
\rm On pose $\displaystyle V=]0,+\infty [\times \left]-\frac{\pi}{2},\frac{\pi}{2}\right[$, et on considère l'application $\Phi : V\to U$ définie par
$$\Phi (r,\theta )=(r\cos \theta,r\sin \theta )$$
\begin{enumerate}
\item L'application $\Phi$ est un ${\mathscr C}^1$-difféomorphisme de $V$ sur $U$, et
$$\forall (x,y)\in U,\; \;\Phi ^{-1}(x,y)=\left( \sqrt{x^2+y^2},\arctan\frac{y}{x}\right).$$
\item Soit $f$ une fonction de classe ${\mathscr C}^1$ solution de $(E)$ sur $U$. On considère la fonction $g$ définie sur $V$ par
$$g(r,\theta )=f(x,y)\mbox{ avec }(x,y)=(r\cos \theta,r\sin \theta ).$$
\begin{enumerate}
\item On exprime les dérivées partielles premières de $f$ en fonction de $g$, $r$ et $\theta$ (cf. les relations $(\star \star )$ ci-dessus).
\item On reporte dans l'équation $(E)$ ce qui donne :
$$r\frac{\partial g}{\partial r}(r,\theta )=r\Leftrightarrow \frac{\partial g}{\partial r}(r,\theta )=1.$$
\item On voit que $g$ est une solution d'une équation du type $(E_1)$, donc $g(r,\theta )=r+k(\theta )$ o\`u $k$ est une fonction de classe ${\mathscr C}^1$ sur $\displaystyle \left]-\frac{\pi}{2},\frac{\pi}{2}\right[$. On en déduit que toute solution $f$ de $(E)$ est de la forme :
$$\displaystyle f(x,y)=\sqrt{x^2+y^2}+k\left(\arctan \frac{y}{x}\right).$$
\end{enumerate}                 
\end{enumerate}

\vskip6mm


\chapter{\bf Intégrales multiples}
\section{\bf Intégrales doubles}
\subsection{Définition via le théorème de Fubini}

\thispagestyle{empty}

\vskip4mm

\noindent Dans cette partie, on ne développe pas toute la théorie des intégrales doubles. Pour calculer de telles intégrales, l'idée est de se ramener à des intégrales simples, que vous savez calculer gr\^ace au théorème fondamental de l'analyse. 

\vskip6mm

\begin{theoreme}[\bf Théorème fondamental de l'analyse]Soit $f:[a,b]\to \Rr$ une fonction continue sur $[a,b]$. Alors $f$ admet une primitive $F$ sur $[a,b]$ et
$$\int_{a}^{b}f(x)\d x=F(b)-F(a).$$
\end{theoreme}

\vskip4mm

\noindent On transforme l'intégrale double en $2$ intégrales simples embo\^{\i}tées. Le théorème clé, qui permet de faire ceci, est le théorème de Fubini. Donnons d'abord la définition suivante :

\vskip6mm

\begin{definition}Soit $D$ un domaine borné de $\Rr^2$.
\begin{enumerate}
\item On dit que $D$ est élémentaire par rapport à $x$ s'il existe $a,b\in \Rr$ et $h_1,h_2:[a,b]\to \Rr$ continues sur $[a,b]$ tels que
$$D=\left\{(x,y)\in \Rr^2\mid a\leq x\leq b,\; h_1(x)\leq y\leq h_2 (x)\right\}.$$
\item On dit que $D$ est élémentaire par rapport à $y$ s'il existe $c,d\in \Rr$ et $k_1,k_2:[c,d]\to \Rr$ continues sur $[c,d]$ tels que
$$D=\left\{(x,y)\in \Rr^2\mid c\leq y\leq d,\; k_1(y)\leq x\leq k_2 (y)\right\}.$$
\end{enumerate} 
\end{definition}

%\centerline{
%\begin{pspicture}(-3,-1)(5,4)
%\psset{unit=0.8}
%\psaxes[axesstyle=axes,labels=none,ticks=none]{->}(0,0)(-0.5,-0.5)(5.5,5)
%\psplot[linewidth=0.6pt,algebraic]{1}{5}{-0.4*x*x+2.5*x+0.5}
%\psplot[linewidth=0.6pt,algebraic]{1}{5}{0.4*x*x-2*x+3}
%\psline[linewidth=0.6pt,linestyle=solid](2.5,4.5)(2.5,0)
%\Rrput(2.5,-0.3){$x$}
%\psline[linewidth=1pt,linestyle=solid](1,2.6)(1,1.4)
%\psline[linewidth=1pt,linestyle=dotted](1,2)(1,0)
%\Rrput(1,-0.3){$a$}
%\psline[linewidth=1pt,linestyle=dotted](5,3)(5,0)
%\Rrput(5,-0.3){$b$}
%\psline[linewidth=1pt,linestyle=dotted](2.5,0.5)(0,0.5)
%\Rrput(-0.6,0.5){$h_1(x)$}
%\psline[linewidth=1pt,linestyle=dotted](2.5,4.2)(0,4.2)
%\Rrput(-0.6,4.2){$h_2(x)$}
%\Rrput(3,-1){$\mbox{\it Domaine élémentaire/}x$}
%\end{pspicture}
%%
%\begin{pspicture}(-2,-1)(5,4)
%\psset{unit=0.8}
%\psaxes[axesstyle=axes,labels=none,ticks=none]{->}(0,0)(-0.3,-0.5)(4.5,5)
%\psarc[linewidth=0.6pt,linestyle=solid]{-}(3.1,2.5){2}{110}{250}
%\psarc[linewidth=0.6pt,linestyle=solid]{-}(1.7,2.5){2}{-70}{70}
%\psline[linewidth=0.6pt,linestyle=solid](4,2)(0,2)
%\Rrput(-0.3,2){$y$}
%\psline[linewidth=1pt,linestyle=dotted](2.4,0.65)(0,0.65)
%\Rrput(-0.3,0.65){$c$}
%\psline[linewidth=1pt,linestyle=dotted](2.4,4.35)(0,4.35)
%\Rrput(-0.3,4.35){$d$}
%\psline[linewidth=1pt,linestyle=dotted](1.2,2)(1.2,0)
%\Rrput(1.2,-0.3){$k_1(y)$}
%\psline[linewidth=1pt,linestyle=dotted](3.6,2)(3.6,0)
%\Rrput(3.6,-0.3){$k_2(y)$}
%\Rrput(2.5,-1){$\mbox{\it Domaine élémentaire/}y$}
%\end{pspicture}
%}
%
%\vskip6mm

\begin{theoreme}[\bf Théorème de Fubini]Soit $D$ un domaine borné de $\Rr^2$ et $f:D\to \Rr$ une fonction continue. Alors $f$ est intégrable sur $D$. De plus, avec les notations précédentes,
\begin{enumerate}
\item si $D$ est élémentaire par rapport à $x$, l'intégrale de $f$ sur $D$ est
$$\int \!\!\!\!\int _Df(x,y)\d x\d y =\int_a^b\left[\int _{h_1 (x)}^{h_2 (x)}f(x,y)\d y\right]\d x,$$
\item si $D$ est élémentaire par rapport à $y$, l'intégrale de $f$ sur $D$ est
$$\int \!\!\!\!\int _Df(x,y)\d x\d y =\int_c^d\left[\int _{k_1 (y)}^{k_2 (y)}f(x,y)\d x\right]\d y.$$
\item Si $D$ est élémentaire à la fois par rapport à $x$ et par rapport à $y$ alors
$$\int \!\!\!\!\int _Df(x,y)\d x\d y =\int_a^b\left[\int _{h_1 (x)}^{h_2 (x)}f(x,y)\d y\right]\d x=\int_c^d\left[\int _{k_1 (y)}^{k_2 (y)}f(x,y)\d x\right]\d y.$$
\end{enumerate} 
\end{theoreme}

\vskip6mm

\noindent{\bf Commentaire. }
\begin{enumerate}
\item En un mot, dans le premier mode, on dit qu'on intègre d'abord par rapport à $y$ puis par rapport à $x$. Et dans le second, on intègre d'abord par rapport à $x$ puis par rapport à $y$.
\item Si $D$ est élémentaire à la fois par rapport à $x$ et par rapport à $y$. On peut appliquer indifférement l'une des deux formules : le calcul est différent, mais le résultat est le m\^eme.
\item Dans la pratique, il peut se faire que l'un des deux modes de calcul soit plus commode que l'autre (cf. exemple 2 ci-dessous).
\end{enumerate}

\vskip6mm

\begin{corollaire}Si $D$ est le rectangle, $D=[a,b]\times [c,d]$, on a :
$$\int \!\!\!\!\int _Df(x,y)\d x\d y =\int_a^b\left[\int _{c}^{d}f(x,y)\d y\right]\d x=\int_c^d\left[\int _{a}^{b}f(x,y)\d x\right]\d y.$$
Si, de plus, $f(x,y)=\varphi (x)\psi (y)$ alors
$$\int \!\!\!\!\int _Df(x,y)\d x\d y =\int_a^b\varphi (x)\d x\int _{c}^{d}\psi (y)\d y.$$
\end{corollaire}

\vskip6mm

\noindent On ramène le calcul d'une intégrale double à celui de deux intégrales simples. L'intégrale double hérite donc de toutes les propriétés de l'intégrale simple.

\vskip6mm

\begin{proposition}[\bf Propriétés]L'intégrale double vérifie les propriétés suivantes :
\begin{enumerate}
\item[$(I_1)$] (Normalisation) Si $f$ et $g$ sont égales, sauf sur un nombre fini de courbes, alors 
$$\displaystyle \int \!\!\!\!\int _{D}f(x,y)\d x\d y=\int \!\!\!\!\int _{D}g(x,y)\d x\d y.$$
\item[$(I_2)$] (Linéarité) Pour tout $\lambda ,\mu \in \Rr$, on a :
$$\int \!\!\!\!\int _{D}(\lambda f+\mu g)\d x\d y=\lambda \int \!\!\!\!\int _{D}f\d x\d y+\mu \int \!\!\!\!\int _{D}g\d x\d y.$$
\item[$(I_4)$] ( Additivité selon le domaine) Si $D=D_1\cup D_2$ tel que $D_1\cap D_2$ est formée d'au plus un nombre fini de courbes, alors
$$\int \!\!\!\!\int _{D}f(x,y)\d x\d y=\int \!\!\!\!\int _{D_1}f(x,y)\d x\d y+\int \!\!\!\!\int _{D_2}f(x,y)\d x\d y.$$
\end{enumerate}
\end{proposition}

\vskip6mm

\noindent{\bf Exemples. }Calculer les intégrales doubles suivantes :
\begin{enumerate}
\item $\displaystyle I=\int \!\!\!\!\int _Dx\e ^{x+y}\d x\d y$, avec $D=[0,1]\times [0,1]$.
On a 
$$\begin{array}{ccl}I&=&\displaystyle \int_0^1\left[\int _{0}^{1}x\e ^{x+y}\d y\right]\d x=\int_0^1\left[x\e ^{x+y}\right]_0^1\d x \\ \\ &=&\displaystyle (\e -1)\int_0^1x\e ^{x}\d x=(\e -1)\left[x\e ^x-\e ^x\right]_0^1=\e -1.\end{array}$$
\item $\displaystyle I=\int \!\!\!\!\int _D\e^{y^2}\d x\d y$, o\`u $D$ est le triangle de sommets $O=(0,0)$, $A=(0,1)$, $B=(1,1)$. Le domaine $D$ est élémentaire à la fois par rapport à $x$ et par rapport à $y$ :
$$\begin{array}{l}\displaystyle D=\left\{(x,y)\in \Rr^2\mid 0\leq x\leq 1,\; x\leq y\leq 1\right\}\\ \\ \displaystyle D=\left\{(x,y)\in \Rr^2\mid 0\leq y\leq 1,\; 0\leq x\leq y\right\}.\end{array}$$
On a :
$$\displaystyle I=\int _0^1\left[\int _x^1\e ^{y^2}\d y\right]\d x =\int _0^1\left[\int _0^y\e ^{y^2}\d x\right]\d y.$$
On ne peut finir le calcul avec de cette formule. On essaie donc avec l'autre mode :
$$\displaystyle I=\int _0^1\left[\int _0^y\e ^{y^2}\d x\right]\d y=\int _0^1y\e ^{y^2}\d y=\left[\frac{\e ^{y^2}}{2}\right]_0^1=\frac{\e -1}{2}.$$
\item $\displaystyle I=\int \!\!\!\!\int _D\d x\d y$ lorsque $D$ est le quadrilatère de sommets $A=(1,0)$, $B=(4,0)$, $C=(2,1)$, $D=(3,1)$. Un dessin donne :
$$D=\left\{(x,y)\in \Rr^2\mid 0\leq y\leq 1,\;y+1\leq x\leq 4-y\right\}.$$
Donc $\displaystyle I=\int_0^1\left[\int _{y+1}^{4-y}\d x\right]\d y=\int_0^1(3-2y)\d y=\left[3y-y^2\right]_0^1=2$.
\item $\displaystyle I=\int \!\!\!\!\int _D(x^2-y^2)\d x\d y$ avec $D=\{(x,y)\in \Rr^2\mid -1\leq x\leq 1,\;|x|\leq y\leq 1\}$. Ici $D=D_1\cup D_2$ avec 
$$\begin{array}{ccl}\displaystyle D_1&=&\{(x,y)\in \Rr^2|\; -x\leq y\leq 1,\; -1\leq x\leq 0\}\\ \\ \displaystyle D_2&=&\{(x,y)\in \Rr^2|\; x\leq y\leq 1,\; 0\leq x\leq 1\}.\end{array}$$
Or
$$\int \!\!\!\!\int _{D_1}(x^2-y^2)\d x\d y=\int_{-1}^0\left[\int _{-x}^{1}(x^2-y^2)\d y\right]\d x=-\frac{1}{6}$$
et 
$$\int \!\!\!\!\int _{D_2}(x^2-y^2)\d x\d y=\int_{0}^1\left[\int _{x}^{1}(x^2-y^2)\d y\right]\d x=-\frac{1}{6}.$$
Donc $\displaystyle I=\int \!\!\!\!\int _{D_1}(x^2-y^2)\d x\d y+\int \!\!\!\!\int _{D_2}(x^2-y^2)\d x\d y=-\frac{1}{3}$.
\end{enumerate}

\vskip6mm

\noindent{\bf Interprétation géométrique. }Soit $D$ un domaine borné de $\Rr^2$.
\begin{enumerate}
\item On suppose que $f:D\to \Rr$ est positive sur $D$. Soit ${\mathscr V}_f$ le domaine dé fini par
$${\mathscr V}_f=\{(x,y,z)\in \Rr^3|\; 0\leq z\leq f(x,y)\mbox{ et }(x,y)\in D\}.$$
C'est la partie de l'espace situé au-dessus de $D$ et en-dessous du graphe de $f$. Alors, par définition le volume du domaine ${\mathscr V}_f$, est
$$V=\int \!\!\!\!\int _Df(x,y)\d x\d y.$$
\item En particulier, si $f$ est constante de valeur $1$, cette intégrale est ègale à l'aire de $D$ :
$$\aire(D)=\int \!\!\!\!\int _D\d x\d y.$$
\end{enumerate}

\vskip8mm

\subsection{Changement de variables}

\vskip4mm

\begin{theoreme}Soient $D$ et $\Delta$ deux domaines bornés de $\Rr^2$ et $\Phi :\Delta \to D$ un $C^1$-difféomorphisme :
$$\Phi :(u,v)\mapsto (x(u,v),y(u,v)).$$
Alors, pour toute fonction $f:D\to \Rr$ continue sur $D$, on a :
$$\int \!\!\!\!\int _{D}f(x,y)\d x\d y=\int \!\!\!\!\int _{\Delta}f\circ \Phi (u,v)\left|\jac _{\Phi}\right|(u,v)\d u\d v.$$
\end{theoreme}

\vskip4mm

\noindent Noter le présence de la \underline{\it valeur absolue} du jacobien de $\Phi$ et que $D=\Phi (\Delta )$ et $\Delta =\Phi ^{-1}(D)$.

\vskip6mm

\noindent $\bullet $ {\bf Changement de variables affine.}

\vskip4mm

\noindent On pose $\left\{\begin{array}{ccc}x&=&au+bv\\y&=&cu+dv\end{array}\right.$ avec $(a,b,c,d)\in \Rr^4$ et $ad-bc\neq 0$. Ici le jacobien est
$$\frac{D(x,y)}{D(u,v)}=\left|\begin{array}{cc}a&b \\ c&d\end{array}\right|=ad-bc.$$

\vskip6mm

\noindent{\bf Exemple. }A l'aide du changement de variables $x=u+v$ et $y=u-v$, calculer $\displaystyle \int \!\!\!\!\int _D\frac{\d x\d y}{1+x+y}$ o\`u  
$$D=\{(x,y)\in \Rr^ 2|\; 0\leq x+y\leq 2,\quad 0\leq x-y\leq 2\}.$$
On a :
$$\frac{D(x,y)}{D(u,v)}=\left|\begin{array}{cc}1&1 \\ 1&-1 \end{array}\right|=-2$$
et
$$(x,y)\in D\Leftrightarrow 0\leq x+y\leq 2\mbox{ et }0\leq x-y\leq 2\Leftrightarrow 0\leq u\leq 1\mbox{ et }0\leq v\leq 1.$$
Donc $(x,y)\in D\Leftrightarrow (u,v)\in \Delta =[0,1]\times [0,1]$. Ainsi
$$\displaystyle I=2\int \!\!\!\!\int _{\Delta}\frac{\d u\d v}{1+2u}=2\left[\int _0^1\d v\right]\left[\int _{0}^{1}\frac{\d u}{1+2u}\right]=\left[\ln (1+2u)\right]_{0}^1=\ln 3.$$

\vskip6mm

\noindent $\bullet $ {\bf Passage en coordonnées polaires.}

\vskip4mm

\noindent Soit $D$ un domaine de $\Rr^2\setminus \{(0,0)\}$. Pour $(x,y)\in D$, on pose 
$$\left\{\begin{array}{ccc}x&=&r\cos \theta \\ y&=&r\sin \theta .\end{array}\right. $$
On a : $\displaystyle \frac{D(x,y)}{D(r,\theta)}=r>0$, et $(x,y)\in D\Leftrightarrow (r,\theta )\in \Delta $. Pour déterminer graphiquement $\Delta$, on dessine le domaine $D$.

%\begin{center}
%\begin{pspicture}(-1,-0.5)(6,4.5)
%\psset{unit=0.8}
%\psaxes[axesstyle=axes,labels=none,ticks=none]{->}(0,0)(-0.2,-0.2)(5.5,5)
%\psellipse[linewidth=0.8pt,fillstyle=hlines,hatchwidth=0.2pt,hatchsep=4pt](3,3)(2,1)
%\psline[linewidth=1.2pt,linestyle=dotted](0,0)(5.5,2.8)
%\psline[linewidth=1.2pt,linestyle=dotted](0,0)(1.5,4.5)
%\psline[linewidth=0.8pt,linestyle=solid](0,0)(4.8,4.8)
%\psarc[linewidth=0.6pt]{->}(0,0){1}{0}{25}
%\Rrput(1.3,0.3){$\theta _1$}
%\psarc[linewidth=0.8pt]{->}(0,0){2.5}{0}{74}
%\Rrput(2.7,0.3){$\theta _2$}
%\psarc[linewidth=1pt]{->}(0,0){1.7}{0}{45}
%\Rrput(1.9,0.35){$\theta$}
%\psdot(2.1,2.1)
%\Rrput(2.1,2.5){$M_1$}
%\psdot(3.9,3.9)
%\Rrput(3.8,4.2){$M_2$}
%\end{pspicture}
%\end{center}

\begin{enumerate}
\item[.] On détermine les bornes de $\theta $. Ce sont $\theta _1,\;\theta _2\in \Rr$ avec $0<\theta _2-\theta _1\leq 2\pi$.
\item[.] Pour $\theta \in ]\theta _1,\theta _2[$, on trace la droite d'angle $\theta $ passant par O. Remarquer les points $M_1$ et $M_2$.
\item[.] On pose $h_1(\theta )=\Vert\overrightarrow{OM_1}\Vert $ et $h_2(\theta )=\Vert\overrightarrow{OM_2}\Vert $.
\end{enumerate}

\vskip4mm

\noindent Ainsi $\displaystyle \Delta =\left\{(r,\theta )\in \Rr^2\mid \theta _1\leq \theta \leq \theta _2,\; h_1(\theta )\leq r\leq h_2(\theta)\right\}$ et
$$\int \!\!\!\!\int _Df(x,y)\d x\d y=\int _{\theta _1}^{\theta _2}\left[\int _{h_1(\theta)}^{h_2(\theta)}f(r\cos \theta,r\sin \theta)r\d r\right]\d \theta.$$

\vskip6mm

\noindent{\bf Exemple 1. }Calculer $\displaystyle I=\int \!\!\!\!\int _Dxy\d x\d y$ avec $D=\left\{(x,y)\in (\Rr^+)^2|x^2+y^2\leq 1\right\}$.

\vskip2mm

\noindent On pose $x=r\cos \theta $ et $y=r\sin \theta $. On a :
$$(x,y)\in D\Leftrightarrow (r,\theta )\in \Delta =[0,1]\times \left[0,\frac{\pi}{2}\right]\quad \mbox{et}\quad \frac{D(x,y)}{D(r,\theta)}=r.$$
Donc
$$I=\int \!\!\!\!\int _{\Delta}r^3\cos \theta \sin \theta \d r\d \theta =\int _0^1r^3\d r\int _0^{\frac{\pi}{2}}\cos \theta \sin \theta \d \theta =\left[\frac{r^4}{4}\right]_0^1\left[\frac{\sin ^2\theta }{2}\right]_0^{\frac{\pi}{2}}=\frac{1}{8}.$$

\vskip4mm

\noindent{\bf Exemple 2. }Calculer $\displaystyle I=\int \!\!\!\!\int _D\frac{x}{x^2+y^2}\d x\d y$ avec $\displaystyle D=\left\{(x,y)\in \Rr^ 2|\frac{1}{2}\leq x,\; x^2+y^2\leq 1\right\}$. On pose $x=r\cos \theta $ et $y=r\sin \theta $, on aura :
$$(x,y)\in D\Leftrightarrow (r,\theta \in \Delta =\left\{(r,\theta )\in \Rr^2\mid -\frac{\pi}{3}\leq \theta \leq \frac{\pi}{3},\; \frac{1}{2\cos \theta}\leq r\leq 1\right\}.$$
D'o\`u
$$I=\int \!\!\!\!\int _{\Delta}\cos \theta \d r\d \theta=\int _{-\frac{\pi}{3}}^{\frac{\pi}{3}}\left[\int _{\frac{1}{2\cos \theta}}^1\cos \theta \d r\right]\d \theta =\int _{-\frac{\pi}{3}}^{\frac{\pi}{3}}\left(\cos \theta -\frac{1}{2}\right)\d \theta =\sqrt{3}-\frac{\pi}{3}.$$

\vskip8mm

\section{\bf Intégrales triples}
\subsection{Théorème de Fubini}

\vskip4mm

\begin{theoreme}Soit $f:D\to \Rr$ une fonction continue sur le domaine borné $D$ de $\Rr^3$. L'intégrale triple de $f$ sur $D$ notée $\displaystyle \int \!\!\!\!\int \!\!\!\!\int_{D}f(x,y,z)\d x\d y\d z$ est définie comme suit.

\vskip2mm

\begin{enumerate}
\item Si l'on peux représenter $D$ sous la forme
$$D=\{(x,y,z)\in \Rr^3|\; (x,y)\in D_0,\; k_1(x,y)\leq z\leq k_2(x,y)\},$$
o\`u $D_0$ est un domaine borné de $\Rr^2$ et $k_1$, $k_2$ sont continues sur $D_0$, alors
$$\int \!\!\!\!\int \!\!\!\!\int _Df(x,y,z)\d x\d y \d z=\int \!\!\!\!\int _{D_0}\left(\int _{k_1 (x,y)}^{k_2 (x,y)}f(x,y,z)\d z\right)\d x\d y.$$

\vskip2mm

\item Si l'on peux représenter $D$ sous la forme
$$D=\{(x,y,z)\in \Rr^3|\; (x,y)\in D(z),\; a\leq z\leq b\},$$
o\`u, pour tout $z\in [a,b]$, $D(z)$ est un domaine borné de $\Rr^2$, alors
$$\int \!\!\!\!\int \!\!\!\!\int _Df(x,y,z)\d x\d y \d z=\int _a^b\left(\int \!\!\!\!\int _{D(z)}f(x,y,z)\d x\d y\right)\d z.$$
\end{enumerate}
\end{theoreme}

\vskip6mm

\noindent{\bf Remarques. }
\begin{enumerate}
\item Le premier cas est appelé sommation par piles, on notera que $D_0$ est la projection orthogonale de $D$ sur le plan $z=0$.
\item Le second cas est appelé sommation par tranches, on notera que $D(z)$ est la section plane de $D$ par le plan de cote $z$.
\item Le volume de $D$, noté $\vol (D)$, se calcule en choisissant $f=1$ :
$$\vol (D)=\int \!\!\!\!\int \!\!\!\!\int _D\d x\d y \d z.$$
\end{enumerate}

\vskip6mm

\begin{corollaire}Si $D=[a_1,a_2]\times [b_1,b_2]\times [c_1,c_2]$ alors 
$$\int \!\!\!\!\int \!\!\!\!\int_{D}f(x,y,z)\d x\d y\d z=\int _{a_1}^{a_2}\left[\int _{b_1}^{b_2}\left(\int _{c_1}^{c_2}f(x,y,z)\d z\right)\d y\right]\d x$$
et on peut permuter l'ordre des intégrations. Si, en plus, $f(x,y,z)=u(x)v(y)w(z)$, alors
$$\int \!\!\!\!\int \!\!\!\!\int_{D}f(x,y,z)\d x\d y\d z=\left[\int _{a_1}^{a_2}u(x)\d x\right]\left[\int _{b_1}^{b_2}v(y)\d y\right]\left[\int _{c_1}^{c_2}w(z)\d z\right].$$
\end{corollaire}

\vskip6mm

\noindent{\bf Exemple 1. }Calculer le volume du tétraèdre $D=\left\{(x,y,z)\in (\Rr^+)^3\mid x+y+z\leq 1\right\}$. On écrit 
$$D=\left\{(x,y,z)\in (\Rr^+)^3\mid (x,y)\in D_0,\; 0\leq z\leq 1-x-y\right\}$$
avec $\displaystyle D_0=\left\{(x,y)\in (\Rr^+)^2\mid x+y\leq 1\right\}$. D'o\`u
$$\begin{array}{ccl}I&=&\displaystyle\int \!\!\!\!\int \!\!\!\!\int_{D}\d x\d y\d z=\int \!\!\!\!\int_{D_0}\left(\int _{0}^{1-x-y}\d z\right)\d x\d y\\ \\ &=&\displaystyle \int _{0}^{1}\left[\int _{0}^{1-x}\left(\int _{0}^{1-x-y}\d z\right)\d y\right]\d x=\frac{1}{6}.\end{array}$$

\vskip6mm

\noindent{\bf Exemple 2. }$\displaystyle I=\int \!\!\!\!\int \!\!\!\!\int_{D}z\d x\d y\d z$, o\`u $\displaystyle D=\{ (x,y,z)\in (\Rr^{+})^3\mid y^{2}+z\leq 1,x^{2}+z\leq 1\} $. 
C'est le second cas, on écrit
$$D=\left\{(x,y,z)\in (\Rr^+)^3\mid 0\leq z\leq 1,\; (x,y)\in D(z)\right\}$$
avec $\displaystyle D(z)=\left\{(x,y)\in \Rr^2\mid 0\leq x\leq \sqrt{1-z},\; 0\leq y\leq \sqrt{1-z}\right\}$. D'o\`u
$$\begin{array}{ccl}I&=&\displaystyle \int _{0}^{1}\left(\int \!\!\!\!\int_{D(z)}z\d x\d y\right)\d z=\int _{0}^{1}\left[z\int _0^{\sqrt{1-z}}\d x\times \int _0^{\sqrt{1-z}}\d y\right]\d z\\ \\ &=&\displaystyle \int _{0}^{1}\left[z(1-z)\right]\d z=\frac{1}{2}-\frac{1}{3}=\frac{1}{6}.\end{array}$$

\vskip6mm

\noindent{\bf Exemple 3. }Calculer $\displaystyle I=\int \!\!\!\!\int \!\!\!\!\int_{D}xyz\d x\d y\d z$, o\`u $D=[0,1]^3$. On applique le corollaire :
$$I=\left[\int _{0}^{1}x\d x\right]\left[\int _{0}^{1}y\d y\right]\left[\int _{0}^{1}z\d z\right]=\frac{1}{8}.$$

\vskip6mm

\noindent{\bf Exemple 4. }Calculer $\displaystyle I=\int \!\!\!\!\int \!\!\!\!\int_{D}x^2y\e^{xyz}\d x\d y\d z$, o\`u $D=[0,1]^3$. On applique à nouveau le corollaire :
$$I=\int _{0}^{1}\left[\int _{0}^{1}\left(\int _{0}^{1}x^2y\e^{xyz}\d z\right)\d y\right]\d x=\e-\frac{5}{2}.$$

\vskip8mm

\subsection{Changement de variables}

\vskip4mm

\begin{theoreme}Soient $D$ et $\Delta$ deux domaines bornés de $\Rr^3$ et $\Phi :\Delta \to D$ un $C^1$-difféomorphisme :
$$\Phi :(u,v,w)\mapsto (x(u,v,w),y(u,v,w),z(u,v,w)).$$
Alors, pour toute fonction $f:D\to \Rr$ continue sur $D$, on a :
$$\int \!\!\!\!\int \!\!\!\!\int _Df(x,y,z)\d x\d y\d z=\int \!\!\!\!\int  \!\!\!\!\int _{\Delta}f(\Phi (u,v,w))\left|\jac _{\Phi}(u,v,w)\right|\d u\d v\d w.$$
\end{theoreme}

\vskip6mm

\noindent Noter que $D=\Phi (\Delta )$ et $\Delta =\Phi ^{-1}(D)$. Voici les cas les plus fréquents en pratique.

\vskip6mm

\noindent{\bf Coordonnées cylindriques. }Un point $M(x,y,z)$ de $\Rr^3$ peut \^etre repéré par un système de coordonnées cylindriques $(r,\theta ,z)\in \Rr^+\times [0,2\pi ]\times \Rr$ reliées au coordonnées cartésiennes par
$$x=r\cos \theta \; ,\qquad y=r\sin \theta \; ,\qquad z=z.$$
Le jacobien du changement de variables est $\displaystyle \frac{D(x,y,z)}{D(r,\theta ,z)}=r$.

\vskip6mm

\noindent{\bf Exemple. }Calculer $\displaystyle I=\int \!\!\!\!\int \!\!\!\!\int _Dz^2\d x\d y\d z$ avec 
$$D=\left\{(x,y,z)\in \Rr^3\mid x^2+y^2\leq R^2,\; 0\leq z\leq h\right\}.$$

\noindent Par passage en cylindriques :
$$\Delta =\left\{(r,\theta ,z)\in \Rr^3\mid 0\leq r\leq R,\; 0\leq \theta \leq 2\pi,\; 0\leq z\leq h\right\}.$$
Donc
$$I=\int \!\!\!\!\int \!\!\!\!\int _{\Delta}z^2r\d r\d \theta \d z=\int _0^Rr\d r\int _0^{2\pi}\d \theta\int _{0}^hz^2\d z=\frac{1}{3}\pi R^2h^3.$$

\vskip6mm

\noindent{\bf Coordonnées sphériques. }Un point $M(x,y,z)$ de $\Rr^3$ peut \^etre repéré par un système de coordonnées sphériques $\displaystyle (r,\theta ,\varphi)\in \Delta =\Rr^+\times [0,2\pi]\times [0,\pi ]$ reliées au coordonnées cartésiennes par
$$x=r\sin \varphi \cos \theta \; ,\qquad y=r\sin \varphi \sin \theta \; ,\qquad z=r\cos \varphi .$$
Le jacobien du changement de variables est $\displaystyle \frac{D(x,y,z)}{D(r,\varphi ,\theta )}=r^2\sin \varphi$.

\vskip6mm

\noindent{\bf Exemple. }Calculer $\displaystyle I=\int \!\!\!\!\int \!\!\!\!\int _D(x^2+y^2+z^2)\d x\d y\d z$ avec 
$$D=\left\{(x,y,z)\in \Rr^3\mid x^2+y^2+z^2\leq 1\right\}.$$

\noindent Par passage en sphériques :
$$\Delta =\left\{(r,\theta ,\varphi)\in \Rr^3\mid 0\leq r\leq 1,\; 0\leq \theta \leq 2\pi,\; 0\leq \varphi \leq \pi\right\}.$$
Donc
$$I=\int \!\!\!\!\int \!\!\!\!\int _{\Delta}r^4\sin \varphi \d r\d \theta \d \varphi =\int _0^1r^4\d r\int _0^{2\pi}\d \theta\int _{0}^{\pi}\sin \varphi \d \varphi =\frac{4\pi}{5}.$$


\chapter{\bf Intégrales curviligne et de surface}
\section{Intégrales curvilignes}

\thispagestyle{empty}

\subsection{Courbes paramétrées}

\vskip4mm

\begin{definition}Une courbe paramétrée de classe $\mathscr{C}^k$, $k\in \N$, de $\Rr^n$ est une application $\gamma :[a,b]\to \Rr^n$ de classe $\mathscr{C}^k$. L'image $\Gamma =\{\gamma (t)\in \Rr^n\mid t\in [a,b]\}$ qui est la courbe géométrique associée à $\gamma $ est souvent confondu avec $\gamma $. On note alors $\Gamma =\left([a,b],\gamma\right)$ et on dit que $\left([a,b],\gamma\right)$ est une paramétrisation de $\Gamma$.
\begin{enumerate}
\item Une telle courbe est dite simple si l'application $\gamma$ est injective.
\item Et elle est dite fermée si $\gamma (a)=\gamma (b)$.
\end{enumerate}
\end{definition}

\vskip4mm

\noindent{\bf Exemples. }
\begin{enumerate}
\item {\bf Segments : }Soient $A=(a_1,a_2)$ et $B=(b_1,b_2)$. Le segment $[AB]$ parcouru de $A$ à $B$ a pour paramétrisation
$$\left\{\begin{array}{ccl}x(t)&=&a_1+t(b_1-a_1)\\ y(t)&=&a_2+t(b_2-a_2),\quad t\in [0,1].\end{array}\right.$$
\item{\bf Cercles : }Le cercle $C_r$ de centre $A=(a,b)$ et de rayon $r>0$ a pour équation : $(x-a)^2+(y-b)^2=r^2$. Une paramétrisation de $C_r$ est donnée par
$$\left\{\begin{array}{ccl}x(t)&=&a+r\cos t\\ y(t)&=&b+r\sin t,\quad t\in [0,2\pi ].\end{array}\right.$$
\item{\bf Ellipses : }L'ellipse d'équation $\displaystyle \frac{x^2}{a^2}+\frac{y^2}{b^2}=1$, $a,b>0$, a pour paramétrisation
$$\left\{\begin{array}{ccl}x(t)&=&a\cos t\\ y(t)&=&b\sin t,\quad t\in [0,2\pi ].\end{array}\right.$$
\end{enumerate}

\vskip4mm

\noindent Lorsque la courbe $\gamma$ est de classe ${\mathscr C}^1$ sur $I$, sa dérivée, aussi connue sous le nom de vecteur vitesse et notée parfois $\overrightarrow{\gamma '}$, est donnée par
$$\forall t\in I,\; \; \gamma '(t)=(x'_1(t),\dots ,x'_n(t)).$$

\vskip4mm

\noindent Pour plus de clarté dans la suite, on se restreint à $n=2$ ou $3$ et on note $x$, $y$ et $z$ les coordonnées.

\vskip8mm

\subsection{Intégrale curviligne d'une fonction}

\vskip4mm

\noindent Sur un axe, l'abscisse $x$ d'un point $M$ est $\displaystyle x=\int _0^x\d u$. Sa valeur absolue est la distance entre l'origine et le point $M$. On souhaite étendre cette notion à une courbe.

\vskip6mm

\begin{definition}[\bf Abscisse curviligne]Soit $\Gamma=([a,b],\gamma)$ une courbe de classe $\mathscr{C}^1$. Le point $A=\gamma (a)$ étant choisi comme origine, on appelle abscisse curviligne du point $M(t)$ la quantité
$$s(t)=\int _a^t\|\gamma '(u)\|\d u.$$
\end{definition}

\vskip4mm

\noindent Remarquer que $\displaystyle \d s=\|\gamma '(t)\|\d t$ et donc, pour tout $t_0\in [a,b]$, $\displaystyle \int _{t_0}^t\d s=s(t)-s(t_0)$. La forme différentielle $\d s$ joue le m\^eme r\^ole dans les intégrales sur une courbe que $\d x$ dans les intégrales sur un axe. Et l'abscisse curviligne joue le m\^eme r\^ole pour une courbe que l'abscisse pour un axe. En particulier, la longueur de la courbe $\Gamma$ est
$$L[\Gamma ]=\int _{\Gamma}\d s=\int _a^b\|\gamma '(t)\|\d t.$$

\vskip6mm

\begin{definition}[\bf Proposition]Soit $\Gamma=([a,b],\gamma)$ une courbe de classe $\mathscr{C}^1$ et $f$ une fonction continue sur $\Gamma$. L'intégrale curviligne de $f$ sur $\Gamma$ est par définition 
$$\int _a^bf\left(\gamma(t)\right)\|\gamma '(t)\|\d t.$$
Cette valeur est indépendante du paramétrage choisi et on la note $\displaystyle \int _{\Gamma}f\d s$.
\end{definition}

\vskip4mm

\noindent{\bf Exemple 1. }Le graphe $\Gamma =\{(x,y)\in \Rr^2\mid x\in [a,b]\mbox{ et }y=f(x)\}$ d'une fonction réelle $f$ a pour paramétrisation
$$\Gamma\,:\,\left\{\begin{array}{l}x(t)=t\\ y(t)=f(t),\quad t\in [a,b]\end{array}\right. \rightarrow \left\{\begin{array}{l}x'(t)=1 \\ y'(t)=f'(t).\end{array}\right.$$
Donc, sur $\Gamma$, $\d s=\sqrt{1+\left[f'(t)\right]^2}\d t$ et la longueur de $\Gamma $ est $\displaystyle L[\Gamma]=\int _a^b\sqrt{1+\left[f'(t)\right]^2}\d t$.

\vskip6mm

\noindent{\bf Exemple 2. }Soit $\Gamma$ le cercle dans le plan $z=1$ de centre $(0,0,1)$ et de rayon $r>0$. Une paramétrisation de $\Gamma$ est 
$$\Gamma\; :\; \left\{\begin{array}{l}x(t)=r\cos t \\ y(t)=r\sin t\\ z(t)=1,\quad t\in [0,2\pi]\end{array}\right. \rightarrow \left\{\begin{array}{l}x'(t)=-r\sin t \\ y'(t)=r\cos t\\ z'(t)=0.\end{array}\right.$$
Ainsi, sur $\Gamma$, $\d s=\sqrt{\left[x'(t)\right]^2+\left[y'(t)\right]^2+\left[z'(t)\right]^2}\d t=r\d t$ et la longueur du cercle est
$$L[\Gamma]=\int _{\Gamma}\d s=\int _0^{2\pi}r\d t=2\pi r.$$
Soit $f(x,y,z)=x^2+y^2+z^2$. Sa restriction à $\Gamma$ est
$$f(r\cos t,r\sin t,1)=r^2\cos ^2t+r^2\sin ^2t+1=r^2+1$$
et son intégrale curviligne sur $\Gamma$ est $\displaystyle \int _{\Gamma}f\d s=\int _0^{2\pi}r(r^2+1)\d t=2\pi r(r^2+1)$.

\vskip8mm

\subsection{Circulation d'un champ de vecteurs}

\vskip4mm

\begin{definition}[\bf Orientation d'une courbe]Orienter une courbe $\Gamma $ c'est choisir un sens de parcours. On note par $\Gamma ^{+}$ la courbe $\Gamma$ lorsqu'on fixe un sens de parcours, et on note par $\Gamma ^{-}$ la m\^eme courbe mais avec le sens de parcours opposé à celui de $\Gamma ^+$. On dit qu'une paramétrisation $([a,b],\gamma )$ de $\Gamma$ est compatible avec $\Gamma ^+$ si le point $\gamma (t)$ se déplace dans le sens de parcours de $\Gamma ^+$ lorsque le paramètre cro\^{\i}t de $a$ à $b$.
\end{definition}

\vskip4mm

\noindent Soit $\Gamma =\left(I,\gamma\right)$ une courbe de classe ${\mathscr C}^1$ régulière, c'est à dire $\gamma '(t)\neq 0$ pour tout $t\in I$. On définit le champ de vecteurs tangents unitaires, noté $\overrightarrow{\tau }$, par
$$\forall t\in I,\; \overrightarrow{\tau }(t)=\frac{\overrightarrow{\gamma '}(t)}{\|\overrightarrow{\gamma '}(t)\|}.$$
Pour tout $t\in I$, $\overrightarrow{\tau }(t)$ est un vecteur directeur unitaire de la tangente à la courbe au point $\gamma (t)$ et il pointe dans le sens de parcours de $\Gamma$.

\vskip6mm

\begin{definition}[\bf Proposition]Soit $D$ un domaine de $\Rr^2$ contenant une courbe $\Gamma$ de paramétrisation $\gamma :t\in [a,b]\to \Gamma$ et $\overrightarrow{V}$ un champ de vecteurs continu sur $D$. La circulation de $\overrightarrow{V}$ sur $\Gamma$ est par définition
$$\int _a^b\overrightarrow{V}(\gamma (t)).\overrightarrow{\gamma '}(t)\d t.$$
Cette valeur est indépendante de toute paramétrisation compatible avec $\Gamma ^+$. On la note $\displaystyle \int _{\Gamma ^+}\overrightarrow{V}.\overrightarrow{\d s}$ o\`u $\overrightarrow{\d s}=\overrightarrow{\tau }\d s$ est le vecteur de l'abscisse curviligne.
\end{definition}

\vskip4mm

\noindent{\bf Exemple 1. }Calculer $\displaystyle \int _{\Gamma }\overrightarrow{V}.\overrightarrow{\d s}$ o\`u $\Gamma =\{(x,y)\in \Rr^2\mid |x|\leq 1 \mbox{ et }y=x^2\}$ allant du point $(-1,1)$ au point $(1,1)$ et $\overrightarrow{V}=y^2\vec{i}+xy\vec{j}$. Une paramétrisation de $\Gamma$ est donnée par
$$\Gamma \,:\,\left\{\begin{array}{ll}x(t)=t &\\ y(t)=t^2,&t\in [-1,1]\end{array}\right. \rightarrow \left\{\begin{array}{l}x'(t)=1 \\ y'(t)=2t.\end{array}\right.$$
D'o\`u 
$$\overrightarrow{V}(\gamma (t))=t^4\vec{i}+t^3\vec{j},\qquad \overrightarrow{V}(\gamma (t)).\overrightarrow{\gamma '}(t)=3t^4$$
et
$$\int _{\Gamma}\overrightarrow{V}.\overrightarrow{\d s}=\int _{-1}^13t^4\d t=\left[\frac{3}{5}t^5\right]_{-1}^1=\frac{6}{5}.$$

\vskip4mm

\noindent{\bf Exemple 2. }Calculer $\displaystyle \int _{\Gamma }\overrightarrow{V}.\overrightarrow{\d s}$ o\`u $\Gamma =\{(x,y)\in \Rr^2\mid x^2+y^2=1,\, 0\leq y\}$ parcouru dans le sens trigonométrique et $\overrightarrow{V}=-y\vec{i}+x\vec{j}$. Une paramétrisation de $\Gamma$ est donnée par
$$\Gamma \,:\,\left\{\begin{array}{ll}x(t)=\cos t &\\ y(t)=\sin t,&t\in [0,\pi]\end{array}\right. \rightarrow \left\{\begin{array}{l}x'(t)=-\sin t \\ y'(t)=\cos t.\end{array}\right.$$
D'o\`u 
$$\overrightarrow{V}(\gamma (t))=-\sin t\vec{i}+\cos t \vec{j},\qquad \overrightarrow{V}(\gamma (t)).\overrightarrow{\gamma '}(t)=\sin ^2t+\cos ^2t=1$$
et
$$\int _{\Gamma}\overrightarrow{V}.\overrightarrow{\d s}=\int _{0}^{\pi}\d t=\pi.$$

\vskip4mm

\begin{proposition}[\bf Propriétés de l'intégrale curviligne]\
\begin{enumerate}
\item Si on change l'orientation, l'intégrale curviligne d'un champ de vecteurs change de signe :
$$\int _{\Gamma ^+}\overrightarrow{V}.\overrightarrow{\d s}=-\int _{\Gamma ^-}\overrightarrow{V}.\overrightarrow{\d s}.$$
\item La relation de Chasles : soit $\Gamma _1$ et $\Gamma _2$ deux courbes dont l'intersection contient au plus un nombre fini de points. Alors 
$$\displaystyle \int _{\Gamma _1^+\cup \Gamma _2^+}\overrightarrow{V}.\overrightarrow{\d s}=\int _{\Gamma _1^+}\overrightarrow{V}.\overrightarrow{\d s}+\int _{\Gamma _2^+}\overrightarrow{V}.\overrightarrow{\d s}.$$
\item Si un champ de vecteurs $\overrightarrow{V}$ dérive d'un potentiel $f$ sur $U$, alors pour toute courbe $\Gamma =([a,b],\gamma )$ de $U$, on a : $\displaystyle \int _{\Gamma ^+}\overrightarrow{V}.\overrightarrow{\d s}=f(\gamma (b))-f(\gamma (a))$.
\end{enumerate}
\end{proposition}

\vskip4mm

\noindent{\it Démonstration.}
\begin{enumerate}
\item Soit $\Gamma ^+$ la courbe orientée par $\gamma :t\in [a,b]\mapsto (x(t,y(t))$. Alors $\Gamma ^-$ est orientée par le paramétrage $\delta :t\in [a,b]\mapsto \gamma(a+b-t)$. On a donc
$$\int _{\Gamma ^-}\overrightarrow{V}.\overrightarrow{\d s}=\int _a^b\left[V_1(\gamma (a+b-t))\delta _1'(t)+V_2(\gamma (a+b-t))\delta _2'(t)\right]\d t$$
On pose $u=a+b-t$. Du fait que $\delta _1'(t)=-x'(a+b-t)$ et $\delta _2'(t)=-y'(a+b-t)$, on obtient :
$$\int _{\Gamma ^-}\overrightarrow{V}.\overrightarrow{\d s}=\int _b^a\left[-V_1(\gamma (u))x'(u)-V_2(\gamma (u))y'(u)\right](-\d u)=-\int _{\Gamma ^+}\overrightarrow{V}.\overrightarrow{\d s}.$$

\item On se ramène à la relation de Chasles classique moyennant le choix d'un paramétrage.

\item On a : $\displaystyle \nabla f.\overrightarrow{\d s}=\left[\frac{\partial f}{\partial x}(\gamma (t))x'(t)+\frac{\partial f}{\partial y}(\gamma (t))x'(t)\right]\d t=\frac{\d (f\circ \gamma )(t)}{\d t}\d t$. Donc
$$\int _{\Gamma }\nabla f.\overrightarrow{\d s}=\int _a^b\frac{\d (f\circ \gamma )(t)}{\d t}\d t=[(f\circ \gamma )(t)]_a^b=f(\gamma (b))-f(\gamma (a))$$
\end{enumerate}

\vskip6mm

\begin{corollaire} Soit $\Gamma$ une courbe \underline{fermée} de $U$, de classe ${\mathscr C}^1$. Si un champ de vecteurs $\overrightarrow{V}$ dérive d'un potentiel sur $U$, alors 
$$\displaystyle \int _{\Gamma}\overrightarrow{V}.\overrightarrow{\d s}=0.$$
Par contraposée, si $\displaystyle \int _{\Gamma}\overrightarrow{V}.\overrightarrow{\d s}\neq 0$, alors $\overrightarrow{V}$ ne dérive pas d'un potentiel sur $U$.
\end{corollaire}

\vskip4mm

\noindent{\bf Exemple. }Soit $\overrightarrow{V}$ le champ de vecteurs défini sur $U=\Rr^2\setminus \{(0,0)\}$ par
$$\overrightarrow{V}=-\frac{y}{x^2+y^2}\vec i+\frac{x}{x^2+y^2}\vec j.$$
On vérifie que $\overrightarrow{V}$ satisfait la condition du théorème de Poincaré pour dériver d'un potentiel :
$$\frac{\partial P}{\partial y}=\frac{\partial Q}{\partial x}=\frac{y^2-x^2}{(x^2+y^2)^2}\rightarrow \mbox{rot}\left(\overrightarrow{V}\right)=0.$$
On calcule la circulation de $\overrightarrow{V}$ sur le cercle unité $C^+$ paramétré comme suit 
$$\Gamma \; :\; \left\{\begin{array}{ll}x(t)=\cos t &\\ y(t)=\sin t&t\in [0,2\pi]\end{array}\right. \rightarrow \left\{\begin{array}{l}x'(t)=-\sin t \\ y'(t)=\cos t.\end{array}\right.$$
On aura $\overrightarrow{V}(x(t),y(t)).\overrightarrow{\gamma '}(t)=\sin ^2t+\cos ^2t=1$, et donc,
$$\displaystyle \int _{C^+}\overrightarrow{V}.\overrightarrow{\d s}=\int _0^{2\pi}dt=2\pi .$$
Donc $\overrightarrow{V}$ ne dérive pas d'un potentiel sur $U$ car sa circulation sur une courbe fermée n'est pas nulle.

\vskip8mm

\subsection{Théorème de Green-Riemann}

\vskip4mm

\noindent{\bf Orientation du bord. }Soit D un domaine de $\Rr^2$ dont le bord est formé d'un nombre $k$ de courbes simples et fermées $C_1,\dots ,C_k$. On oriente son bord suivant la convention de la matière à gauche : $"$lorsque l'on parcourt n'importe qu'elle courbe $C_i$ du bord on doit avoir le domaine $D$ sur sa gauche$"$. On dit que le bord est orienté dans le sens direct.

%\begin{center}
%\begin{pspicture}(-3,-2)(3,2.2)
%\psset{unit=1}
%\psarc[linewidth=0.5pt,arrowsize=6pt]{->}(0,0){2}{90}{92}
%\psarc[linewidth=0.5pt,arrowsize=6pt]{<-}(0,0){0.6}{91}{90}
%\pscircle[linestyle=solid,linewidth=0.5pt](0,0){2}
%\pscircle[linestyle=solid,linewidth=0.5pt,fillstyle=hlines,fillsep=0.5,hatchwidth=0.3pt](0,0){0.6}
%\Rrput(-1.2,0.6){$D$}
%\Rrput(1.732,1){$\bullet$}
%\psline[linewidth=0.5pt]{->}(1.732,1)(2.6,1.5)
%\Rrput(2.5,1.2){$\vec{n}$}
%\psline[linewidth=0.5pt]{->}(1.732,1)(1.232,1.866)
%\Rrput(1.55,1.8){$\vec{\tau}$}
%\psline[linewidth=0.5pt](1.89,1.12)(1.8,1.3)(1.66,1.2)
%\end{pspicture}
%\end{center}

\begin{theoreme}[\bf Green-Riemann] Soit $D\subset \Rr^2$ un domaine borné dont le bord $\Gamma $ est une réunion de courbes fermées simples de classe ${\mathscr C}^1$ par morceaux et orientée dans le sens direct. Soit $\overrightarrow{V}=P\vec i+Q\vec j$ un champ de vecteurs de classe ${\mathscr C}^1$ sur $D$. Alors
$$\int _{\Gamma ^+}\overrightarrow{V}.\overrightarrow{\d s}=\int \!\!\!\!\int _D\Rrot \left(\overrightarrow{V}\right)\d x\d y=\int \!\!\!\!\int _D\left(\frac{\partial Q}{\partial x}-\frac{\partial P}{\partial y}\right)\d x\d y.$$
\end{theoreme}

\vskip4mm

\noindent{\bf Exemples. }
\begin{enumerate}
\item Calculons la circulation du champ $\overrightarrow{V}(x,y)=(y+3x,2y-x)$ le long de l'ellipse $\Gamma$ : $4x^2+y^2=4$ parcourue dans le sens trigonométrique. Posons $P(x,y)= y+3x$ et $Q(x,y)=2y-x$. Ainsi 
$$\frac{\partial P}{\partial y}=1\quad \mbox{et}\quad \frac{\partial Q}{\partial x}=-1$$
$P$ et $Q$ étant de classe ${\mathscr C}^1$ à l'intérieur de l'ellipse, on peut donc utiliser la formule de Green-Riemann :
$$\int _{\Gamma}\overrightarrow{V}.\overrightarrow{\d s}=\int \!\!\!\!\int _K\left(\frac{\partial Q}{\partial x}-\frac{\partial P}{\partial y}\right)\d x\d y=-2\int \!\!\!\!\int _K\d x\d y=-2\mbox{Aire}(K)=-4\pi.$$
\item Calculons la circulation du champ $\overrightarrow{V}(x,y)=(3xy,x^2)$ le long de la frontière $\Gamma $ du rectangle $R=[-1,3]\times [0,2]$ parcourue dans le sens trigonométrique.

\vskip2mm

\noindent Les fonctions $P$ et $Q$ sont de classe ${\mathscr C}^1$ à l'intérieur du rectangle, on peut donc appliquer la formule de Green-Riemann :
$$\int _{\Gamma}\overrightarrow{V}.\overrightarrow{\d s}=\int \!\!\!\!\int _R(-x)\d x\d y=\int _{-1}^3(-x)\d x\times \int _0^2\d y=-8.$$
\end{enumerate}

\vskip8mm

\section{Intégrales de surface}
\subsection{Surfaces paramétrées}

\vskip4mm

\begin{definition}Une surface paramétrée de classe $\mathscr{C}^1$ est une application 
$$\begin{array}{ccccl}\varphi &:&D&\to &\Rr^3\\ &&(u,v)&\mapsto &(\varphi _1(u,v),\varphi _2(u,v),\varphi _3(u,v)\end{array}$$
de classe $\mathscr{C}^1$ sur un domaine $D$ de $\Rr^2$. L'ensemble $S=\varphi (D)$ est la surface géométrique associée à $\varphi $. On le notera $S=(D,\varphi)$ et on dira que $\left(D,\varphi\right)$ est une paramétrisation de $S$. On notera aussi $M(u,v)$ le point image de $(u,v)$. Enfin, on dira que $S$ est simple si $\varphi$ est injective.
\end{definition}

\vskip4mm

\begin{definition}On dit que le point $M(u,v)=\varphi (u,v)$ de la surface $S=(D,\varphi)$ est régulier si la jacobienne de $\varphi$ au point $(u,v)$ est de rang $2$ et on dit aussi que la surface $S$ est lisse si tout point $M$ de $S$ est régulier.
\end{definition}

\vskip4mm

\noindent Soit $S=(D,\varphi)$ une surface paramétrée. On notera $\displaystyle \frac{\partial \varphi}{\partial u}$ et $\displaystyle \frac{\partial \varphi}{\partial v}$ les champs de vecteurs définis par
$$\displaystyle \frac{\partial \varphi}{\partial u}=\left(\frac{\partial \varphi _1}{\partial u},\frac{\partial \varphi _2}{\partial u},\frac{\partial \varphi _3}{\partial u}\right)$$
et
$$\displaystyle \frac{\partial \varphi}{\partial v}=\left(\frac{\partial \varphi _1}{\partial v},\frac{\partial \varphi _2}{\partial v},\frac{\partial \varphi _3}{\partial v}\right).$$
Le point $M(u_0,v_0)$ est régulier si, et seulement si, les vecteurs $\displaystyle \frac{\partial \varphi}{\partial u}(u_0,v_0)$ et $\displaystyle \frac{\partial \varphi}{\partial v}(u_0,v_0)$ ne sont pas colinéaires, ou encore, si, et seulement si, leur produit vectoriel
$$\overrightarrow{N}(u_0,v_0)=\frac{\partial \varphi}{\partial u}(u_0,v_0)\wedge \frac{\partial \varphi}{\partial v}(u_0,v_0)$$
est non nul.

\vskip6mm

\begin{proposition}[\bf Définition]Soit $M_0(u_0,v_0)$ un point régulier de la surface $S=(D,\varphi)$. Le plan tangent à la surface $S$ au point $M_0$ est le plan passant par $M_0$ et de vecteurs directeurs
$$\frac{\partial \varphi}{\partial u}(u_0,v_0)\quad \mbox{et}\quad \frac{\partial \varphi}{\partial v}(u_0,v_0).$$
C'est aussi le plan passant par $M_0$ et de vecteur normal $\overrightarrow{N}(u_0,v_0)$.
\end{proposition}

\vskip4mm

\noindent{\bf Remarque. }Le vecteur $\displaystyle \overrightarrow{n}(u_0,v_0)=\frac{\overrightarrow{N}(u_0,v_0)}{\|\overrightarrow{N}(u_0,v_0)\|}$ est un vecteur unitaire normal au plan tangent à la surface $S$ au point $M_0$. Ce vecteur nous permet d'orienter la surface $S$.

\vskip6mm

\noindent{\bf Exemple. }La shpère $S_r=\{(x,y,z)\in \Rr^3\mid x^2+y^2+z^2=r^2\}$ de centre $O$ et de rayon $r$ est paramétrée par $\displaystyle \Psi :[0,\pi ]\times [0,2\pi]\to \Rr^3$ avec
$$\Psi (\varphi ,\theta )=\left(r\sin \varphi \cos \theta \, ,\, r\sin \varphi \sin \theta  \, ,\, r\cos \varphi \right).$$
Les vecteurs directeurs de l'espace tangent au point $M(\varphi ,\theta )$ sont
$$\left\{\begin{array}{l}\displaystyle \frac{\partial \Psi}{\partial \varphi}(\varphi ,\theta )=\left(r\cos \varphi \cos \theta \, ,\, r\cos \varphi\sin \theta \, ,\, -r\sin \varphi\right)\\ \\ \displaystyle \frac{\partial \Psi}{\partial \theta}(\varphi ,\theta )=\left(-r\sin \varphi\sin \theta \, ,\, r\sin \varphi\cos \theta \, ,\, 0\right).\end{array}\right.$$
Le vecteur normal $\overrightarrow{N}$ au point $M(\varphi ,\theta )$ est le produit vectoriel de ces deux vecteurs :
$$\overrightarrow{N}(\varphi ,\theta )=r^2\sin \varphi \left(\sin \varphi \cos \theta \, ,\, \sin \varphi \sin \theta \, ,\, \cos \varphi \right).$$

\vskip8mm

\subsection{Intégrale de surface d'une fonction}

\vskip4mm

\noindent Rappelons que l'aire d'un parallélogramme de c\^oté $\overrightarrow{AB}$ et $\overrightarrow{AC}$ est $\|\overrightarrow{AB}\wedge \overrightarrow{AC}\|$. Soit $S=(D,\varphi)$ une surface paramétrée de classe $\mathscr{C}^1$. Supposons que les paramètres $u$ et $v$ varient de $\d u$ et $\d v$ ; le point $M(u,v)$ décrit le parallélogramme de c\^otés $\displaystyle \d u\frac{\partial \varphi }{\partial u}$ et $\displaystyle \d v\frac{\partial \varphi }{\partial v}$. L'aire de ce parallélogramme est l'élément d'aire :
$$\d \sigma =\left\|\frac{\partial \varphi }{\partial u}\wedge \frac{\partial \varphi }{\partial v}\right\|\d u\d v.$$
Remarquer que l'élément d'aire s'écrit aussi : $\d \sigma =\|\overrightarrow{N}\|\d u\d v$.

\vskip6mm

\begin{definition}[\bf Proposition]Soit $S=(D,\varphi)$ une surface de classe $\mathscr{C}^1$ et $f$ une fonction continue sur $S$. L'intégrale de surface de $f$ sur $S$ est par définition 
$$\int \!\!\!\!\int _Df\left(\varphi (u,v)\right)\left\|\frac{\partial \varphi }{\partial u}\wedge \frac{\partial \varphi }{\partial v}\right\|\d u\d v.$$
Cette valeur est indépendante du paramétrage choisi et on la note $\displaystyle \int \!\!\!\!\int _Sf\d \sigma$.
\end{definition}

\vskip4mm

\noindent Par définition, l'aire de $S$ est obtenue en prenant pour $f$ la fonction constante égale à $1$ :
$$\aire (S)=\int \!\!\!\!\int _S\d \sigma=\int \!\!\!\!\int _D\left\|\frac{\partial \varphi }{\partial u}\wedge \frac{\partial \varphi }{\partial v}\right\|\d u\d v.$$

\vskip6mm

\noindent{\bf Exemple 1. }Calculer l'aire de la calotte sphérique $S_{\alpha}=\{M(\varphi ,\theta )\in S_r\mid 0\leq \varphi \leq \alpha \}$. Une parmétrisation de $S_{\alpha}$ est donnée ci-dessus. Pour $(\varphi ,\theta )\in \Delta =[0,\alpha]\times [0,2\pi]$, le vecteur normal au point $M(\theta ,\varphi)$ est
$$\overrightarrow{N}(\varphi ,\theta )=r^2\sin \varphi \left(\sin \varphi \cos \theta \, ,\, \sin \varphi \sin \theta \, ,\, \cos \varphi \right).$$
Sa norme est $\|\overrightarrow{N}(\varphi ,\theta )\|=r^2\sin \varphi $. Donc $\d \sigma =r^2\sin \varphi \d \varphi \d \theta$ et l'aire de $S_{\alpha}$ est
$$\aire (S_{\alpha})=\int \!\!\!\!\int _{\Delta}r^2\sin \varphi \d \varphi \d \theta=r^2\int _0^{\alpha }\sin \varphi \d \varphi \int _0^{2\pi}\d \theta =2\pi r^2(1-\cos \alpha ).$$
En particulier, pour $\alpha =\pi$, $S_{\alpha}$ est la sphère $S_r$ et son aire est $4\pi r^2$.

\vskip6mm

\noindent{\bf Exemple 2. }L'aire de la surface $\displaystyle S=\left\{(x,y,f(x,y))\in \Rr^3\mid (x,y)\in D\right\}$, o\`u $D$ est un domaine de $\Rr^2$ et $f$ est une fonction de $\mathscr{C}^1$ sur $D$ est
$$\aire (S)=\int\!\!\!\!\int _D\sqrt{1+\left(\frac{\partial f}{\partial x}\right)^2+\left( \frac{\partial f}{\partial y}\right)^2}\d x\d y.$$

\vskip2mm

\noindent En effet, une paramétrisation de la surface $S$ est donnée par 
$$\varphi (x,y)=\left(x,y,f(x,y)\right)\;\; \rightarrow \frac{\partial \varphi}{\partial x}=\left(1,0,\frac{\partial f}{\partial x}\right)\;\; \mbox{et}\;\; \frac{\partial \varphi}{\partial y}=\left(0,1,\frac{\partial f}{\partial y}\right).$$
Or
$$\overrightarrow{N}=\frac{\partial \varphi}{\partial x}\wedge \frac{\partial \varphi}{\partial y}=\left(-\frac{\partial f}{\partial x},-\frac{\partial f}{\partial y},1\right).$$
Donc $\displaystyle \d \sigma =\|\overrightarrow{N}\|\d x\d y=\sqrt{1+\left(\frac{\partial f}{\partial x}\right)^2+\left( \frac{\partial f}{\partial y}\right)^2}\d x\d y$.

\vskip8mm

\subsection{Flux d'un champ de vecteurs}

\vskip4mm

\noindent Pour définir le flux d'un champ de vecteurs à travers une surface de $\Rr^3$, on aura besoin du concept de surface orientable. Une surface de $\Rr^3$ est dite orientable si elle comporte deux c\^otés distincts. Un des deux c\^otés est dit extérieur et l'autre intérieur. Lorsque la notion d'extérieur et d'intérieur n'est pas clair, on précisera quel c\^oté est quoi.

\vskip6mm

\noindent En chaque point régulier d'une telle surface, il existe deux vecteurs unitaires normaux $\overrightarrow{n}$ et $-\overrightarrow{n}$. L'un est dirigé vers l'extérieur de la surface et donc l'autre (qui est son opposé) vers l'intérieur de la surface. Le choix d'un de ces vecteurs oriente la surface.

\vskip6mm

\noindent{\bf Exemple. }La sphère, le cylindre, le tore dans $\Rr^3$ sont des surfaces orientables. Le ruban de M\"obius dans $\Rr^3$ est l'exemple type de surface non-orientable.

\vskip6mm

\begin{definition}[\bf Proposition]Soit $S=(D,\varphi)$ une surface dans $\Rr^3$ et $\overrightarrow{F}$ un champ de vecteurs de $\Rr^3$ défini et continu sur $S$. On note $\overrightarrow{n}$ le vecteur normal unitaire dirigé vers l'extérieur. Le flux de $\overrightarrow{F}$ à travers la surface $S$ est par définition
$$\int \!\!\!\!\int _D\overrightarrow{F}(\varphi(u,v)).\overrightarrow{n}\left(\varphi (u,v)\right)\left\|\frac{\partial \varphi }{\partial u}\wedge \frac{\partial \varphi }{\partial v}\right\|\d u\d v.$$
Cette valeur est indépendante du paramétrage choisi à condition de respecter l'orientation et on la note
$$\int \!\!\!\!\int _S\overrightarrow{F}.\overrightarrow{n}\d \sigma .$$
\end{definition}

\vskip4mm

\noindent{\bf Remarque. }Si on change d'orientation pour la surface, le flux est changé en son opposé.

\vskip6mm

\noindent{\bf Exemple. }Calculer le flux du champ $\overrightarrow{F}(x,y,z)=(x,y,0)$ à travers la shère $S_r$. Une paramétrisation de $S_r$ est donnée par $\displaystyle \Psi :D=[0,\pi ]\times [0,2\pi]\to \Rr^3$ avec
$$\Psi (\varphi ,\theta )=\left(r\sin \varphi \cos \theta \, ,\, r\sin \varphi \sin \theta  \, ,\, r\cos \varphi \right).$$
Le vecteur normal $\overrightarrow{N}$ au point $M(\varphi ,\theta)$
$$\overrightarrow{N}(\varphi ,\theta )=r^2\sin \varphi \left(\sin \varphi \cos \theta \, ,\, \sin \varphi \sin \theta \, ,\, \cos \varphi \right)=r\sin \varphi \Psi (\varphi ,\theta )$$
est non nul sur $]0,\pi [\times [0,2\pi]$ et le vecteur normal unitaire $\displaystyle \overrightarrow{n}=\frac{1}{r}\Psi (\varphi ,\theta )$ est bien dirigé vers l'extérieur. Ainsi $\overrightarrow{F}.\overrightarrow{n}=r\sin ^2\varphi$ et, comme $\d \sigma =r^2\sin \varphi \d \varphi \d \theta$, alors
$$\int \!\!\!\!\int _{S_r}\overrightarrow{F}.\overrightarrow{n}\d \sigma =\int \!\!\!\!\int _Dr^3\sin ^3\varphi \d \varphi \d \theta=r^3\int _0^{2\pi}\d \theta \int _0^{\pi}\sin ^3\varphi \d \varphi =\frac{8\pi r^3}{3}.$$

\vskip8mm

\subsection{Théorèmes Stokes, Ostrogradsky}

\vskip4mm

\begin{definition}Une surface admissible pour le théorème de Stokes, est une surface orientable $S$ de $\Rr^3$ dont le bord $\Gamma =\partial S$ est réunion finie de courbes fermés de $\Rr^3$ orientées de sorte à ce que la surface $S$ soit à gauche.
\end{definition}

\vskip4mm

\noindent La formule du rotationnel ci-dessous relie l'intégrale curviligne d'un champ de vecteur sur un courbe fermé avec le flux du rotationel du m\^eme champ à travers la surface limitée par cette courbe. C'est une traduction de la formule de Green-Riemann pour une courbe fermée dans $\Rr^3$.

\vskip6mm

\begin{theoreme}[\bf Stokes]Soit $S$ une surface admissible pour le théorème de Stokes. Soit $\vec{F}: U\to \Rr^3$ un champ de vecteurs de classe $\mathscr{C}^1$, défini sur un ouvert $U$ contenant $S$. Alors la circulation de $\vec{F}$ sur le bord $\Gamma$ est égale au flux du rotationnel de $\vec{F}$ à travers la surface $S$ :
$$\oint _{\Gamma}\overrightarrow{F}.\overrightarrow{\d s}=\int \!\!\!\!\int _S\overrightarrow{\mathrm rot}(\overrightarrow{F}).\overrightarrow{n}\d \sigma .$$
\end{theoreme}

\vskip6mm

\noindent{\bf Rappel. }Si la courbe $\Gamma$ est paramétrée par $t\in [a,b]\mapsto (\gamma _1(t),\gamma _2(t),\gamma _3(t))$ alors 
$$\overrightarrow{F}.\overrightarrow{\d s}=\left[F_1(\gamma (t))\gamma _1'(t)+F_2(\gamma (t))\gamma _2'(t)+F_2(\gamma (t))\gamma _2'(t)\right]\d t$$
et
$$\oint _{\Gamma}\overrightarrow{F}.\overrightarrow{\d s}=\int _a^b\left[F_1(\gamma (t))\gamma _1'(t)+F_2(\gamma (t))\gamma _2'(t)+F_2(\gamma (t))\gamma _2'(t)\right]\d t.$$

\vskip6mm

\noindent{\bf Exemple. }Vérifier le théorème de Stokes lorsque $\overrightarrow{F}$ est le champ de vecteurs de $\Rr^3$ défini sur $\Rr^3$ par
$$F(x,y,z)=(z-y,x+z,-(x+y))$$
et $S$ est la parabolo\"{\i}de $S=\{(x,y,z)\in \Rr^3\mid z=4-x^2-y^2\; ,\; 0\leq z\leq 4\}$.

\vskip3mm

\noindent Le théorème de Stokes s'applique car $S$ est une surface admissible pour ce théorème et $\overrightarrow{F}$ est un champ de vecteurs de classe $\mathscr{C}^1$ sur tout $\Rr^3$.

\vskip3mm

\noindent Le bord $\partial S$ de $S$ est le cercle $C$ de rayon $2$ dans le plan $z=0$. Ainsi, une paramétrisation régulière de $C$, telle
que la surface $S$ soit à gauche, est 
$$\gamma \,:\,t\in [0,2\pi]\mapsto (2\cos t,2\sin t,0)\; \rightarrow \; \gamma'(t)=(-2\sin t,2\cos t,0).$$
La circulation de $\overrightarrow{F}$ sur $C$ est
$$\int _{C}\overrightarrow{V}.\overrightarrow{\d M}=\int _{0}^{2\pi}\overrightarrow{F}(\gamma (t)).\overrightarrow{\gamma '(t)}\d t=\int _{0}^{2\pi}4\d t=8\pi.$$
Calculons maintenant le membre de gauche de la formule de Stokes. Une paramétrisation
régulière de $S$, telle que le vecteur normal pointe vers l'extérieur, est donnée par 
$$\begin{array}{ccccl}\Psi &:&[0,2]\times [0,2\pi]&\to &\Rr^3\\ &&(r,\theta )&\mapsto & (r\cos \theta \, ,\, r\sin \theta  \, ,\, 4-r^2).\end{array}$$
Le vecteur normal de cette paramétrisation est
$$\overrightarrow{N}(r,\theta)=\left(\begin{array}{c}\cos \theta \\ \sin \theta \\ -2r\end{array}\right)\wedge \left(\begin{array}{c}-r\sin \theta \\ r\cos \theta \\ 0 \end{array}\right)=\left(\begin{array}{c}2r^2\cos \theta \\ 2r^2\sin \theta \\ r \end{array}\right).$$
Il est bien dirigé vers l'extérieur. Un calcul immédiat donne $\mathrm{rot}(\overrightarrow{F})=(-2,2,2)$. Ainsi
$$\int \!\!\!\!\int _S\mathrm{rot}(\overrightarrow{F}).\overrightarrow{n}\d \sigma =\int\!\!\!\!\int _{\Delta}\mathrm{rot}(\overrightarrow{F})\left(\Psi (r,\theta)\right).\overrightarrow{N}\left(\Psi (r,\theta)\right)\d r\d \theta $$
o\`u $\Delta =[0,2]\times [0,2\pi]$. D'après Fubini, on obtient :
$$\int \!\!\!\!\int _S\mathrm{rot}(\overrightarrow{F}).\overrightarrow{n}\d \sigma =\int _0^2\left[\int _0^{2\pi}2r\left(2r(-\cos \theta +\sin \theta )+1\right)\d \theta\right]\d r =4\pi \int _0^2r\d r=8\pi.$$

\vskip6mm

\noindent La formule de la divergence relie le flux d'un champ de vecteurs à travers une surface fermée à l'intégrale triple de la divergence de ce champ sur le domaine de $\Rr^3$ limité par cette surface.

\vskip6mm

\begin{theoreme}[\bf Ostrogradsky]Soit $\overrightarrow{F}: U\to \Rr^3$ un champ de vecteurs de classe $\mathscr{C}^1$, défini sur un ouvert $U$ contenant la surface $S$ et le volume $V$ qu'elle limite. On suppose que $S$ est orientée de sorte que le vecteur normal unitaire soit dirigé vers l'extérieur. Alors
$$\int \!\!\!\!\int _S\overrightarrow{F}.\overrightarrow{n}\d \sigma =\int \!\!\!\!\int \!\!\!\!\int _V\mathrm{div}(\overrightarrow{F})\d x\d y\d z.$$
\end{theoreme}

\vskip6mm

\noindent{\bf Remarque. }
\begin{enumerate}
\item Ce théorème s'appelle aussi théorème de la divergence de Gauss, ou théorème de la divergence de Gauss-Ostrogradsky.
\item Le terme de droite est une intégrale de Riemann triple. Elle est bien définie car l'intégrant est continu sur son domaine de définition, et le domaine d'intégration est admissible au sens de Riemann.
\item Ce théorème est utilisé pour calculer l'intégrale de surface d'un champ de vecteurs car l'intégrale de Riemann est plus facile à évaluer.
\end{enumerate}

\vskip6mm

\noindent{\bf Exemple. }Calculer le flux du champ de vecteurs $\overrightarrow{F}(x,y,z) =(x,y,z)$ à travers le bord du cube centré à l'origine et de c\^oté de longueur $2$.

\vskip3mm

\noindent Le théorème de la divergence s'applique car le cube $V$ est un domaine admissible pour ce théorème et le champ $\overrightarrow{F}$ est de classe $\mathscr{C}^1$ sur tout $\Rr^3$. Or $\mathrm{div}(\overrightarrow{F})=3$. Donc, d'après Fubini,
$$\int \!\!\!\!\int _{\partial V}\overrightarrow{F}.\overrightarrow{n}\d \sigma =\int \!\!\!\!\int \!\!\!\!\int _V3\d x\d y\d z=3\left[\int _{-1}^1\d x\right]\times \left[\int _{-1}^1\d y\right]\times \left[\int _{-1}^1\d z\right]=24.$$

\vskip6mm

\begin{corollaire}Soit $S_1$ et $S_2$ deux surfaces fermées telles que $S_2$ est dans l'intérieur de $S_1$. Soit $\overrightarrow{F}$ un champ de vecteurs de classe $\mathscr{C}^1$ tel que $\mathrm{div}(\overrightarrow{F})\equiv 0$ sur un ouvert contenant le volume entre les deux surfaces. Alors
$$\int \!\!\!\!\int _{S_1}\overrightarrow{F}.\overrightarrow{n}\d \sigma =-\int \!\!\!\!\int _{S_2}\overrightarrow{F}.\overrightarrow{n}\d \sigma .$$
\end{corollaire}


\end{document}




